% Options for packages loaded elsewhere
\PassOptionsToPackage{unicode}{hyperref}
\PassOptionsToPackage{hyphens}{url}
%
\documentclass[
]{article}
\usepackage{lmodern}
\usepackage{amssymb,amsmath}
\usepackage{ifxetex,ifluatex}
\ifnum 0\ifxetex 1\fi\ifluatex 1\fi=0 % if pdftex
  \usepackage[T1]{fontenc}
  \usepackage[utf8]{inputenc}
  \usepackage{textcomp} % provide euro and other symbols
\else % if luatex or xetex
  \usepackage{unicode-math}
  \defaultfontfeatures{Scale=MatchLowercase}
  \defaultfontfeatures[\rmfamily]{Ligatures=TeX,Scale=1}
\fi
% Use upquote if available, for straight quotes in verbatim environments
\IfFileExists{upquote.sty}{\usepackage{upquote}}{}
\IfFileExists{microtype.sty}{% use microtype if available
  \usepackage[]{microtype}
  \UseMicrotypeSet[protrusion]{basicmath} % disable protrusion for tt fonts
}{}
\makeatletter
\@ifundefined{KOMAClassName}{% if non-KOMA class
  \IfFileExists{parskip.sty}{%
    \usepackage{parskip}
  }{% else
    \setlength{\parindent}{0pt}
    \setlength{\parskip}{6pt plus 2pt minus 1pt}}
}{% if KOMA class
  \KOMAoptions{parskip=half}}
\makeatother
\usepackage{xcolor}
\IfFileExists{xurl.sty}{\usepackage{xurl}}{} % add URL line breaks if available
\IfFileExists{bookmark.sty}{\usepackage{bookmark}}{\usepackage{hyperref}}
\hypersetup{
  pdftitle={NETCapstone},
  pdfauthor={Netaverner},
  hidelinks,
  pdfcreator={LaTeX via pandoc}}
\urlstyle{same} % disable monospaced font for URLs
\usepackage[margin=1in]{geometry}
\usepackage{color}
\usepackage{fancyvrb}
\newcommand{\VerbBar}{|}
\newcommand{\VERB}{\Verb[commandchars=\\\{\}]}
\DefineVerbatimEnvironment{Highlighting}{Verbatim}{commandchars=\\\{\}}
% Add ',fontsize=\small' for more characters per line
\usepackage{framed}
\definecolor{shadecolor}{RGB}{248,248,248}
\newenvironment{Shaded}{\begin{snugshade}}{\end{snugshade}}
\newcommand{\AlertTok}[1]{\textcolor[rgb]{0.94,0.16,0.16}{#1}}
\newcommand{\AnnotationTok}[1]{\textcolor[rgb]{0.56,0.35,0.01}{\textbf{\textit{#1}}}}
\newcommand{\AttributeTok}[1]{\textcolor[rgb]{0.77,0.63,0.00}{#1}}
\newcommand{\BaseNTok}[1]{\textcolor[rgb]{0.00,0.00,0.81}{#1}}
\newcommand{\BuiltInTok}[1]{#1}
\newcommand{\CharTok}[1]{\textcolor[rgb]{0.31,0.60,0.02}{#1}}
\newcommand{\CommentTok}[1]{\textcolor[rgb]{0.56,0.35,0.01}{\textit{#1}}}
\newcommand{\CommentVarTok}[1]{\textcolor[rgb]{0.56,0.35,0.01}{\textbf{\textit{#1}}}}
\newcommand{\ConstantTok}[1]{\textcolor[rgb]{0.00,0.00,0.00}{#1}}
\newcommand{\ControlFlowTok}[1]{\textcolor[rgb]{0.13,0.29,0.53}{\textbf{#1}}}
\newcommand{\DataTypeTok}[1]{\textcolor[rgb]{0.13,0.29,0.53}{#1}}
\newcommand{\DecValTok}[1]{\textcolor[rgb]{0.00,0.00,0.81}{#1}}
\newcommand{\DocumentationTok}[1]{\textcolor[rgb]{0.56,0.35,0.01}{\textbf{\textit{#1}}}}
\newcommand{\ErrorTok}[1]{\textcolor[rgb]{0.64,0.00,0.00}{\textbf{#1}}}
\newcommand{\ExtensionTok}[1]{#1}
\newcommand{\FloatTok}[1]{\textcolor[rgb]{0.00,0.00,0.81}{#1}}
\newcommand{\FunctionTok}[1]{\textcolor[rgb]{0.00,0.00,0.00}{#1}}
\newcommand{\ImportTok}[1]{#1}
\newcommand{\InformationTok}[1]{\textcolor[rgb]{0.56,0.35,0.01}{\textbf{\textit{#1}}}}
\newcommand{\KeywordTok}[1]{\textcolor[rgb]{0.13,0.29,0.53}{\textbf{#1}}}
\newcommand{\NormalTok}[1]{#1}
\newcommand{\OperatorTok}[1]{\textcolor[rgb]{0.81,0.36,0.00}{\textbf{#1}}}
\newcommand{\OtherTok}[1]{\textcolor[rgb]{0.56,0.35,0.01}{#1}}
\newcommand{\PreprocessorTok}[1]{\textcolor[rgb]{0.56,0.35,0.01}{\textit{#1}}}
\newcommand{\RegionMarkerTok}[1]{#1}
\newcommand{\SpecialCharTok}[1]{\textcolor[rgb]{0.00,0.00,0.00}{#1}}
\newcommand{\SpecialStringTok}[1]{\textcolor[rgb]{0.31,0.60,0.02}{#1}}
\newcommand{\StringTok}[1]{\textcolor[rgb]{0.31,0.60,0.02}{#1}}
\newcommand{\VariableTok}[1]{\textcolor[rgb]{0.00,0.00,0.00}{#1}}
\newcommand{\VerbatimStringTok}[1]{\textcolor[rgb]{0.31,0.60,0.02}{#1}}
\newcommand{\WarningTok}[1]{\textcolor[rgb]{0.56,0.35,0.01}{\textbf{\textit{#1}}}}
\usepackage{graphicx,grffile}
\makeatletter
\def\maxwidth{\ifdim\Gin@nat@width>\linewidth\linewidth\else\Gin@nat@width\fi}
\def\maxheight{\ifdim\Gin@nat@height>\textheight\textheight\else\Gin@nat@height\fi}
\makeatother
% Scale images if necessary, so that they will not overflow the page
% margins by default, and it is still possible to overwrite the defaults
% using explicit options in \includegraphics[width, height, ...]{}
\setkeys{Gin}{width=\maxwidth,height=\maxheight,keepaspectratio}
% Set default figure placement to htbp
\makeatletter
\def\fps@figure{htbp}
\makeatother
\setlength{\emergencystretch}{3em} % prevent overfull lines
\providecommand{\tightlist}{%
  \setlength{\itemsep}{0pt}\setlength{\parskip}{0pt}}
\setcounter{secnumdepth}{-\maxdimen} % remove section numbering
\usepackage{booktabs}
\usepackage{longtable}
\usepackage{array}
\usepackage{multirow}
\usepackage{wrapfig}
\usepackage{float}
\usepackage{colortbl}
\usepackage{pdflscape}
\usepackage{tabu}
\usepackage{threeparttable}
\usepackage{threeparttablex}
\usepackage[normalem]{ulem}
\usepackage{makecell}
\usepackage{xcolor}

\title{NETCapstone}
\author{Netaverner}
\date{13/12/2020}

\begin{document}
\maketitle

\hypertarget{overview}{%
\section{Overview}\label{overview}}

This project is related to the Choose-your-own project of the HervardX:
PH125.9x Data Science: Capstone course.

The report starts by giving a general idea of the project. It then shows
how the Dataset will be prepared and setup. An exploratory data analysis
is carried out on the Dataset. This is performed in order to help
facilitate the develop of a machine learning algorithm(s) that can
predict whether the Biomechanical features of orthopedic patients is
Abnormal or Normal. The results of the models created will be examined
and explaing. Lastly, the report have some concluding remarks and ideas
for future work.

\hypertarget{introduction}{%
\section{Introduction}\label{introduction}}

This project focuses on a Biomechanical features of orthopedic patients
Dataset aqquired from kaggle ({\textbf{???}}). This data contains
patients that have medical diagnosis that covers, Disk Hernia and
Spondylolisthesis.

Spondylolisthesis is a spinal condition that affects the lower vertebrae
(spinal bones). This disease causes one of the lower vertebrae to slip
forward onto the bone directly beneath it. It's a painful condition but
treatable in most cases (Moore 2018).

A herniated disk refers to a problem with one of the rubbery cushions
(disks) that sit between the individual bones (vertebrae) that stack to
make your spine (Staff, n.d.).

With the medical field introducing more and more machinery and computer
oriented technology to perform diagnostics on the human body. This
project tries to show that by using machine learning one should be able
to having computers or other technology performing diagnosis on humans,
without the need for human intervention. This could possibly help human
doctors in the future as machines could possible achieve a medical
diagnosis more rapidly, if provided with the needed inputs.

This project will make a performance comparison between different
machine learning algorithms in order to to assess the correctness in
classifying data with respect to efficiency and effectiveness of each
algorithm in terms of accuracy, precision, sensitivity and specificity,
in order to find the best class of patient.

The major models used and tested will be supervised learning models
(algorithms that learn from labeled data), which are generally used in
these kinds of data analysis.

\hypertarget{aim}{%
\subsection{Aim}\label{aim}}

The objective of this report is to train machine learning models to
predict whether Biomechanical features of a orthopedic patient is
Abnormal or Normal. Data will be transformed and its dimension reduced
to reveal patterns in the Dataset and create a more robust analysis.

The optimal model will be selected by selecting the Model that produces
the best results in the following categories: * accuracy * sensitivity *
f1 score

Other factors will also be reviewed when select the optimal model.

Though the use of machine learning method the features of orthopedic
patients will be extracted and classified. The goal is determine whether
a given sample of patients have Normal or Abnormal features.

The machine learning models in this report try to create a classifier
that provides a high accuracy level combined with a low rate of
false-negatives (high sensitivity).

\hypertarget{dataset}{%
\subsection{Dataset}\label{dataset}}

The report covers the Biomechanical features of orthopedic patients
Dataset acquired from
(\url{https://www.kaggle.com/uciml/biomechanical-features-of-orthopedic-patients})
and created and maintained by UCI Machine Learning (Learning, n.d.).

This report focuses on the .csv file ``column2Cweka.csv (file with two
class labels)'' this file contains the following:

The categories Disk Hernia and Spondylolisthesis were merged into a
single category labelled as `abnormal'. Thus, task consists in
classifying patients as belonging to one out of two categories: Normal
(100 patients) or Abnormal (210 patients).

The .csv format file containing the data is loaded from my personal
computer, should one wish to run the .Rmd file please download the
``column2Cweka.csv'' from the URL displayed earlier and load it.

\begin{Shaded}
\begin{Highlighting}[]
\CommentTok{#Libraries}
\ControlFlowTok{if}\NormalTok{(}\OperatorTok{!}\KeywordTok{require}\NormalTok{(dplyr)) }\KeywordTok{install.packages}\NormalTok{(}\StringTok{"dplyr"}\NormalTok{, }\DataTypeTok{repos =} \StringTok{"http://cran.us.r-project.org"}\NormalTok{)}
\ControlFlowTok{if}\NormalTok{(}\OperatorTok{!}\KeywordTok{require}\NormalTok{(ggplot2)) }\KeywordTok{install.packages}\NormalTok{(}\StringTok{"ggplot2"}\NormalTok{, }\DataTypeTok{repos =} \StringTok{"http://cran.us.r-project.org"}\NormalTok{)}
\ControlFlowTok{if}\NormalTok{(}\OperatorTok{!}\KeywordTok{require}\NormalTok{(corrplot)) }\KeywordTok{install.packages}\NormalTok{(}\StringTok{"recorrplotadr"}\NormalTok{, }\DataTypeTok{repos =} \StringTok{"http://cran.us.r-project.org"}\NormalTok{)}
\ControlFlowTok{if}\NormalTok{(}\OperatorTok{!}\KeywordTok{require}\NormalTok{(gridExtra)) }\KeywordTok{install.packages}\NormalTok{(}\StringTok{"gridExtra"}\NormalTok{, }\DataTypeTok{repos =} \StringTok{"http://cran.us.r-project.org"}\NormalTok{)}
\ControlFlowTok{if}\NormalTok{(}\OperatorTok{!}\KeywordTok{require}\NormalTok{(pROC)) }\KeywordTok{install.packages}\NormalTok{(}\StringTok{"pROC"}\NormalTok{, }\DataTypeTok{repos =} \StringTok{"http://cran.us.r-project.org"}\NormalTok{)}
\ControlFlowTok{if}\NormalTok{(}\OperatorTok{!}\KeywordTok{require}\NormalTok{(caTools)) }\KeywordTok{install.packages}\NormalTok{(}\StringTok{"caTools"}\NormalTok{, }\DataTypeTok{repos =} \StringTok{"http://cran.us.r-project.org"}\NormalTok{)}
\ControlFlowTok{if}\NormalTok{(}\OperatorTok{!}\KeywordTok{require}\NormalTok{(caretEnsemble)) }\KeywordTok{install.packages}\NormalTok{(}\StringTok{"caretEnsemble"}\NormalTok{, }\DataTypeTok{repos =} \StringTok{"http://cran.us.r-project.org"}\NormalTok{)}
\ControlFlowTok{if}\NormalTok{(}\OperatorTok{!}\KeywordTok{require}\NormalTok{(grid)) }\KeywordTok{install.packages}\NormalTok{(}\StringTok{"grid"}\NormalTok{, }\DataTypeTok{repos =} \StringTok{"http://cran.us.r-project.org"}\NormalTok{)}
\ControlFlowTok{if}\NormalTok{(}\OperatorTok{!}\KeywordTok{require}\NormalTok{(readr)) }\KeywordTok{install.packages}\NormalTok{(}\StringTok{"readr"}\NormalTok{, }\DataTypeTok{repos =} \StringTok{"http://cran.us.r-project.org"}\NormalTok{)}
\ControlFlowTok{if}\NormalTok{(}\OperatorTok{!}\KeywordTok{require}\NormalTok{(tidyverse)) }\KeywordTok{install.packages}\NormalTok{(}\StringTok{"tidyverse"}\NormalTok{, }\DataTypeTok{repos =} \StringTok{"http://cran.us.r-project.org"}\NormalTok{)}
\ControlFlowTok{if}\NormalTok{(}\OperatorTok{!}\KeywordTok{require}\NormalTok{(caret)) }\KeywordTok{install.packages}\NormalTok{(}\StringTok{"caret"}\NormalTok{, }\DataTypeTok{repos =} \StringTok{"http://cran.us.r-project.org"}\NormalTok{)}
\ControlFlowTok{if}\NormalTok{(}\OperatorTok{!}\KeywordTok{require}\NormalTok{(ggfortify)) }\KeywordTok{install.packages}\NormalTok{(}\StringTok{"ggfortify"}\NormalTok{, }\DataTypeTok{repos =} \StringTok{"http://cran.us.r-project.org"}\NormalTok{)}
\ControlFlowTok{if}\NormalTok{(}\OperatorTok{!}\KeywordTok{require}\NormalTok{(glmnet)) }\KeywordTok{install.packages}\NormalTok{(}\StringTok{"glmnet"}\NormalTok{, }\DataTypeTok{repos =} \StringTok{"http://cran.us.r-project.org"}\NormalTok{)}
\ControlFlowTok{if}\NormalTok{(}\OperatorTok{!}\KeywordTok{require}\NormalTok{(randomForest)) }\KeywordTok{install.packages}\NormalTok{(}\StringTok{"randomForest"}\NormalTok{, }\DataTypeTok{repos =} \StringTok{"http://cran.us.r-project.org"}\NormalTok{)}
\ControlFlowTok{if}\NormalTok{(}\OperatorTok{!}\KeywordTok{require}\NormalTok{(nnet)) }\KeywordTok{install.packages}\NormalTok{(}\StringTok{"nnet"}\NormalTok{, }\DataTypeTok{repos =} \StringTok{"http://cran.us.r-project.org"}\NormalTok{)}
\ControlFlowTok{if}\NormalTok{(}\OperatorTok{!}\KeywordTok{require}\NormalTok{(funModeling)) }\KeywordTok{install.packages}\NormalTok{(}\StringTok{"funModeling"}\NormalTok{, }\DataTypeTok{repos =} \StringTok{"http://cran.us.r-project.org"}\NormalTok{)}
\ControlFlowTok{if}\NormalTok{(}\OperatorTok{!}\KeywordTok{require}\NormalTok{(RefManageR)) }\KeywordTok{install.packages}\NormalTok{(}\StringTok{"RefManageR"}\NormalTok{, }\DataTypeTok{repos =} \StringTok{"http://cran.us.r-project.org"}\NormalTok{)}
\ControlFlowTok{if}\NormalTok{(}\OperatorTok{!}\KeywordTok{require}\NormalTok{(kableExtra)) }\KeywordTok{install.packages}\NormalTok{(}\StringTok{"kableExtra"}\NormalTok{, }\DataTypeTok{repos =} \StringTok{"http://cran.us.r-project.org"}\NormalTok{)}

\CommentTok{# The data file will be loaded from my personal computer}
\CommentTok{#Please add own .csv below if wanting to recreate}
\NormalTok{data <-}\StringTok{ }\KeywordTok{read.csv}\NormalTok{(}\StringTok{"/Users/user/Documents/DataScience/FinalCapstone/NETCapstone/DataFiles/column_2C_weka.csv"}\NormalTok{)}
\end{Highlighting}
\end{Shaded}

\newpage

\hypertarget{methods-and-analysis}{%
\section{Methods And Analysis}\label{methods-and-analysis}}

\hypertarget{data-analysis}{%
\subsection{Data Analysis}\label{data-analysis}}

From observation of the Dataset, one can determine that the Dateset
consists of 310 observations and 7 variables

\begin{Shaded}
\begin{Highlighting}[]
\KeywordTok{str}\NormalTok{(data)}
\end{Highlighting}
\end{Shaded}

\begin{verbatim}
## 'data.frame':    310 obs. of  7 variables:
##  $ pelvic_incidence        : num  63 39.1 68.8 69.3 49.7 ...
##  $ pelvic_tilt.numeric     : num  22.55 10.06 22.22 24.65 9.65 ...
##  $ lumbar_lordosis_angle   : num  39.6 25 50.1 44.3 28.3 ...
##  $ sacral_slope            : num  40.5 29 46.6 44.6 40.1 ...
##  $ pelvic_radius           : num  98.7 114.4 106 101.9 108.2 ...
##  $ degree_spondylolisthesis: num  -0.254 4.564 -3.53 11.212 7.919 ...
##  $ class                   : chr  "Abnormal" "Abnormal" "Abnormal" "Abnormal" ...
\end{verbatim}

\begin{Shaded}
\begin{Highlighting}[]
\KeywordTok{head}\NormalTok{(data)}
\end{Highlighting}
\end{Shaded}

\begin{verbatim}
##   pelvic_incidence pelvic_tilt.numeric lumbar_lordosis_angle sacral_slope
## 1         63.02782           22.552586              39.60912     40.47523
## 2         39.05695           10.060991              25.01538     28.99596
## 3         68.83202           22.218482              50.09219     46.61354
## 4         69.29701           24.652878              44.31124     44.64413
## 5         49.71286            9.652075              28.31741     40.06078
## 6         40.25020           13.921907              25.12495     26.32829
##   pelvic_radius degree_spondylolisthesis    class
## 1      98.67292                -0.254400 Abnormal
## 2     114.40543                 4.564259 Abnormal
## 3     105.98514                -3.530317 Abnormal
## 4     101.86850                11.211523 Abnormal
## 5     108.16872                 7.918501 Abnormal
## 6     130.32787                 2.230652 Abnormal
\end{verbatim}

\begin{Shaded}
\begin{Highlighting}[]
\KeywordTok{summary}\NormalTok{(data)}
\end{Highlighting}
\end{Shaded}

\begin{verbatim}
##  pelvic_incidence pelvic_tilt.numeric lumbar_lordosis_angle  sacral_slope   
##  Min.   : 26.15   Min.   :-6.555      Min.   : 14.00        Min.   : 13.37  
##  1st Qu.: 46.43   1st Qu.:10.667      1st Qu.: 37.00        1st Qu.: 33.35  
##  Median : 58.69   Median :16.358      Median : 49.56        Median : 42.40  
##  Mean   : 60.50   Mean   :17.543      Mean   : 51.93        Mean   : 42.95  
##  3rd Qu.: 72.88   3rd Qu.:22.120      3rd Qu.: 63.00        3rd Qu.: 52.70  
##  Max.   :129.83   Max.   :49.432      Max.   :125.74        Max.   :121.43  
##  pelvic_radius    degree_spondylolisthesis    class          
##  Min.   : 70.08   Min.   :-11.058          Length:310        
##  1st Qu.:110.71   1st Qu.:  1.604          Class :character  
##  Median :118.27   Median : 11.768          Mode  :character  
##  Mean   :117.92   Mean   : 26.297                            
##  3rd Qu.:125.47   3rd Qu.: 41.287                            
##  Max.   :163.07   Max.   :418.543
\end{verbatim}

One needs to determine if the Dataset contains any missing values

\begin{verbatim}
## $pelvic_incidence
## [1] 0
## 
## $pelvic_tilt.numeric
## [1] 0
## 
## $lumbar_lordosis_angle
## [1] 0
## 
## $sacral_slope
## [1] 0
## 
## $pelvic_radius
## [1] 0
## 
## $degree_spondylolisthesis
## [1] 0
## 
## $class
## [1] 0
\end{verbatim}

There are no NA values in this dataset, but there is an unbalance
between the datasets proportions :

\begin{Shaded}
\begin{Highlighting}[]
\CommentTok{#check proportions of income groups}
\KeywordTok{prop.table}\NormalTok{(}\KeywordTok{table}\NormalTok{(data}\OperatorTok{$}\NormalTok{class))}
\end{Highlighting}
\end{Shaded}

\begin{verbatim}
## 
##  Abnormal    Normal 
## 0.6774194 0.3225806
\end{verbatim}

The graph below gives and indication to this disproportion in the target
variable class:

\begin{Shaded}
\begin{Highlighting}[]
\CommentTok{#plot proportion}
\KeywordTok{options}\NormalTok{(}\DataTypeTok{repr.plot.width=}\DecValTok{4}\NormalTok{, }\DataTypeTok{repr.plot.height=}\DecValTok{4}\NormalTok{)}

\KeywordTok{ggplot}\NormalTok{(data, }\KeywordTok{aes}\NormalTok{(class))}\OperatorTok{+}
\StringTok{  }\KeywordTok{geom_bar}\NormalTok{(}\DataTypeTok{fill=}\StringTok{"blue"}\NormalTok{,}\DataTypeTok{alpha=}\FloatTok{0.8}\NormalTok{)}\OperatorTok{+}
\StringTok{  }\KeywordTok{theme_bw}\NormalTok{() }\OperatorTok{+}
\StringTok{  }\KeywordTok{labs}\NormalTok{(}\DataTypeTok{title=}\StringTok{"Distribution of Class"}\NormalTok{) }\OperatorTok{+}
\StringTok{  }\KeywordTok{theme}\NormalTok{(}\DataTypeTok{plot.title =} \KeywordTok{element_text}\NormalTok{(}\DataTypeTok{hjust =} \FloatTok{0.5}\NormalTok{))}
\end{Highlighting}
\end{Shaded}

\includegraphics{NETCapstone_files/figure-latex/unnamed-chunk-7-1.pdf}

Most variables in the dataset are normally distributed as shown in the
below plot, except for degree\_spondylolisthesis:

\includegraphics{NETCapstone_files/figure-latex/unnamed-chunk-8-1.pdf}

\hypertarget{correlations}{%
\subsubsection{Correlations}\label{correlations}}

Check correlation

\begin{Shaded}
\begin{Highlighting}[]
\NormalTok{correlationMatrix <-}\StringTok{ }\KeywordTok{cor}\NormalTok{(data[,}\DecValTok{0}\OperatorTok{:}\DecValTok{6}\NormalTok{])}
\NormalTok{correlationMatrix }\OperatorTok\StringTok{  }
\StringTok{  }\NormalTok{knitr}\OperatorTok{::}\KeywordTok{kable}\NormalTok{() }\OperatorTok\StringTok{ }\KeywordTok{kable_styling}\NormalTok{(}\DataTypeTok{latex_options =} \KeywordTok{c}\NormalTok{(}\StringTok{"striped"}\NormalTok{, }\StringTok{"scale_down"}\NormalTok{))}
\end{Highlighting}
\end{Shaded}

\begin{table}[H]
\centering
\resizebox{\linewidth}{!}{
\begin{tabular}{l|r|r|r|r|r|r}
\hline
  & pelvic\_incidence & pelvic\_tilt.numeric & lumbar\_lordosis\_angle & sacral\_slope & pelvic\_radius & degree\_spondylolisthesis\\
\hline
\cellcolor{gray!6}{pelvic\_incidence} & \cellcolor{gray!6}{1.0000000} & \cellcolor{gray!6}{0.6291988} & \cellcolor{gray!6}{0.7172824} & \cellcolor{gray!6}{0.8149600} & \cellcolor{gray!6}{-0.2474672} & \cellcolor{gray!6}{0.6387427}\\
\hline
pelvic\_tilt.numeric & 0.6291988 & 1.0000000 & 0.4327639 & 0.0623453 & 0.0326678 & 0.3978623\\
\hline
\cellcolor{gray!6}{lumbar\_lordosis\_angle} & \cellcolor{gray!6}{0.7172824} & \cellcolor{gray!6}{0.4327639} & \cellcolor{gray!6}{1.0000000} & \cellcolor{gray!6}{0.5983869} & \cellcolor{gray!6}{-0.0803436} & \cellcolor{gray!6}{0.5336670}\\
\hline
sacral\_slope & 0.8149600 & 0.0623453 & 0.5983869 & 1.0000000 & -0.3421283 & 0.5235575\\
\hline
\cellcolor{gray!6}{pelvic\_radius} & \cellcolor{gray!6}{-0.2474672} & \cellcolor{gray!6}{0.0326678} & \cellcolor{gray!6}{-0.0803436} & \cellcolor{gray!6}{-0.3421283} & \cellcolor{gray!6}{1.0000000} & \cellcolor{gray!6}{-0.0260650}\\
\hline
degree\_spondylolisthesis & 0.6387427 & 0.3978623 & 0.5336670 & 0.5235575 & -0.0260650 & 1.0000000\\
\hline
\end{tabular}}
\end{table}

\begin{Shaded}
\begin{Highlighting}[]
\KeywordTok{corrplot}\NormalTok{(correlationMatrix, }\DataTypeTok{order =} \StringTok{"hclust"}\NormalTok{, }\DataTypeTok{tl.cex =} \DecValTok{1}\NormalTok{, }\DataTypeTok{addrect =} \DecValTok{3}\NormalTok{)  }
\end{Highlighting}
\end{Shaded}

\includegraphics{NETCapstone_files/figure-latex/unnamed-chunk-10-1.pdf}

As seen in plot there seems to be three variables that are highly
correlated with each other (cor \textgreater= 0.7) (University, n.d.) ,
these three variables are pelvic\_incidence ,lumbar\_lordosis\_angle and
sacral\_slope. Due to this we can assume that methods that usually fail
due to high correlation variable maybe be impacted on badly by the
current variables. Thus the highly correlated variable will be removed
as to much correlation can cause some machine learning models to fail.

The Caret R package provides the ``findCorrelation'', which analyses the
correlation matrix of a data's attributes, it then reports on which
attributes can be removed.

Following method below proves the assumption of no highly correlated
variables:

\begin{Shaded}
\begin{Highlighting}[]
\CommentTok{# find Variables that are highly corrected (>0.7)}
\NormalTok{highlyCorrelated <-}\StringTok{ }\KeywordTok{findCorrelation}\NormalTok{(correlationMatrix, }\DataTypeTok{cutoff=}\FloatTok{0.7}\NormalTok{)}
\CommentTok{# print indexes of highly correlated attributes}
\KeywordTok{print}\NormalTok{(highlyCorrelated)}
\end{Highlighting}
\end{Shaded}

\begin{verbatim}
## [1] 1
\end{verbatim}

There is one variables to be removed.

\hypertarget{remove-highly-correlated-variables}{%
\subsubsection{Remove highly correlated
variables}\label{remove-highly-correlated-variables}}

By carefully selecting features in ones data can mean the difference
between poor performance with long training times and great performance
with short training times.

The highly correlated variable is removed as shown in the code below:

\begin{Shaded}
\begin{Highlighting}[]
\NormalTok{data2 <-}\StringTok{ }\NormalTok{data }\OperatorTok\StringTok{ }\KeywordTok{select}\NormalTok{(}\OperatorTok{!}\NormalTok{highlyCorrelated)}
\CommentTok{# number of columns after removing correlated variables}
\KeywordTok{ncol}\NormalTok{(data2)}
\end{Highlighting}
\end{Shaded}

\begin{verbatim}
## [1] 6
\end{verbatim}

One variable is lost, this being pelvic\_incidence. By removing this one
variable the data is cleaned from highly correlated variables. Hense
forth the dataset with the highly correlated variable removed will be
refered to as the Cleaned Dataset

\newpage

\hypertarget{modeling-approach}{%
\section{MODELING Approach}\label{modeling-approach}}

\hypertarget{principal-component-analysis-pca}{%
\subsection{Principal Component Analysis
{[}PCA{]}}\label{principal-component-analysis-pca}}

For this project the function ``prncomp'' will be used to calculate the
PCA, this function is chosen as to avoid relecancy and redundancy. The
``prncomp'' function helps to select components that will avoid
correlated variables. This is importance as mentioned before highly
correlated variable can cause problems with clustering analysis.

PCA is usually used with data that contains a large number of variables.
Although this particular dataset has at max 7 variables the PCA
technique will be used on the dataset as to see how it performs. PCA
works by reduces the dimensions of the feature space by feature
extraction.

PCA is used to reduce the dimensionality of a dataset which consists of
a large number of variables correlated with each other. The variables
can be either heavily or lightly correlated. The reduction of
dimensionality must be done while retaining the variation present in the
dataset, up to the maximum extent.

The same is done by transforming the variables to a new set of
variables, which are known as the principal components {[}PCs{]}. Pcs
are orthogonal and are ordered in such a way that the retention of
variation present in the original variables decreases when moving down
the order. By transforming variables in this mannger, the 1st principal
component retains maximum variation that was present in the original
components. The principal components are the eigenvectors of a
covariance matrix, and are therefore are orthogonal.

It is important to note that the dataset on which PCA technique is used
on, must be scaled. The results are also sensitive to the relative
scaling. (ProjectPro n.d. )

\hypertarget{pca-on-original-data}{%
\subsection{PCA on original data}\label{pca-on-original-data}}

\begin{Shaded}
\begin{Highlighting}[]
\NormalTok{PCAData <-}\StringTok{ }\KeywordTok{prcomp}\NormalTok{(data[,}\DecValTok{0}\OperatorTok{:}\DecValTok{6}\NormalTok{], }\DataTypeTok{center =} \OtherTok{TRUE}\NormalTok{, }\DataTypeTok{scale =} \OtherTok{TRUE}\NormalTok{)}
\KeywordTok{plot}\NormalTok{(PCAData, }\DataTypeTok{type=}\StringTok{"l"}\NormalTok{)}
\end{Highlighting}
\end{Shaded}

\includegraphics{NETCapstone_files/figure-latex/unnamed-chunk-13-1.pdf}

\begin{Shaded}
\begin{Highlighting}[]
\CommentTok{#summary}
\KeywordTok{summary}\NormalTok{(PCAData)}
\end{Highlighting}
\end{Shaded}

\begin{verbatim}
## Importance of components:
##                          PC1    PC2    PC3     PC4     PC5       PC6
## Standard deviation     1.802 1.0930 0.8724 0.68741 0.57098 1.935e-10
## Proportion of Variance 0.541 0.1991 0.1268 0.07875 0.05434 0.000e+00
## Cumulative Proportion  0.541 0.7401 0.8669 0.94566 1.00000 1.000e+00
\end{verbatim}

As seen in the table above the first component can explain 0.541 of the
variance after applying 4 PCs, 0.94566 of the variance can be explained.
According to the summary above, 1.0 of the variance can be explained
after 5 PCS, only a small number of PCs are required as the dataset has
so few variables

\hypertarget{plot-of-pc1-vs-pc2}{%
\subsubsection{Plot of PC1 vs PC2}\label{plot-of-pc1-vs-pc2}}

\begin{Shaded}
\begin{Highlighting}[]
\NormalTok{pcaDf <-}\StringTok{ }\KeywordTok{as.data.frame}\NormalTok{(PCAData}\OperatorTok{$}\NormalTok{x)}
\KeywordTok{ggplot}\NormalTok{(pcaDf, }\KeywordTok{aes}\NormalTok{(}\DataTypeTok{x=}\NormalTok{PC1, }\DataTypeTok{y=}\NormalTok{PC2, }\DataTypeTok{col=}\NormalTok{data}\OperatorTok{$}\NormalTok{class)) }\OperatorTok{+}\StringTok{ }\KeywordTok{geom_point}\NormalTok{(}\DataTypeTok{alpha=}\FloatTok{0.5}\NormalTok{)}
\end{Highlighting}
\end{Shaded}

\includegraphics{NETCapstone_files/figure-latex/unnamed-chunk-15-1.pdf}

From the plot above it can be determined that the first two components
somewhat separated into two classes. This is caused by the fact that the
variance explained by these components is not large.

\hypertarget{plot-of-densitys}{%
\subsubsection{Plot of densitys}\label{plot-of-densitys}}

\begin{Shaded}
\begin{Highlighting}[]
\NormalTok{pc1 <-}\StringTok{ }\KeywordTok{ggplot}\NormalTok{(pcaDf, }\KeywordTok{aes}\NormalTok{(}\DataTypeTok{x=}\NormalTok{PC1, }\DataTypeTok{fill=}\NormalTok{data}\OperatorTok{$}\NormalTok{class)) }\OperatorTok{+}\StringTok{ }\KeywordTok{geom_density}\NormalTok{(}\DataTypeTok{alpha=}\FloatTok{0.25}\NormalTok{)  }
\NormalTok{pc2 <-}\StringTok{ }\KeywordTok{ggplot}\NormalTok{(pcaDf, }\KeywordTok{aes}\NormalTok{(}\DataTypeTok{x=}\NormalTok{PC2, }\DataTypeTok{fill=}\NormalTok{data}\OperatorTok{$}\NormalTok{class)) }\OperatorTok{+}\StringTok{ }\KeywordTok{geom_density}\NormalTok{(}\DataTypeTok{alpha=}\FloatTok{0.25}\NormalTok{)  }

\KeywordTok{grid.arrange}\NormalTok{(pc1, pc2, }\DataTypeTok{ncol=}\DecValTok{2}\NormalTok{)}
\end{Highlighting}
\end{Shaded}

\includegraphics{NETCapstone_files/figure-latex/unnamed-chunk-16-1.pdf}

\hypertarget{pca-cleaned-dataset}{%
\subsection{PCA Cleaned Dataset}\label{pca-cleaned-dataset}}

\begin{Shaded}
\begin{Highlighting}[]
\NormalTok{PCAData2 <-}\StringTok{ }\KeywordTok{prcomp}\NormalTok{(data2[,}\DecValTok{0}\OperatorTok{:}\DecValTok{5}\NormalTok{], }\DataTypeTok{center =} \OtherTok{TRUE}\NormalTok{, }\DataTypeTok{scale =} \OtherTok{TRUE}\NormalTok{)}
\KeywordTok{plot}\NormalTok{(PCAData2, }\DataTypeTok{type=}\StringTok{"l"}\NormalTok{)}
\end{Highlighting}
\end{Shaded}

\includegraphics{NETCapstone_files/figure-latex/unnamed-chunk-17-1.pdf}

\begin{Shaded}
\begin{Highlighting}[]
\KeywordTok{summary}\NormalTok{(PCAData2)}
\end{Highlighting}
\end{Shaded}

\begin{verbatim}
## Importance of components:
##                           PC1    PC2    PC3     PC4     PC5
## Standard deviation     1.5325 1.0930 0.8672 0.68395 0.48679
## Proportion of Variance 0.4697 0.2389 0.1504 0.09356 0.04739
## Cumulative Proportion  0.4697 0.7086 0.8590 0.95261 1.00000
\end{verbatim}

The above table shows that 0.95261 of the variance is explained with 4
PCs in the transformed dataset data2. There does not seem to be must
change with having removed the highly correlated variable but the is
some change in the amount of variance explained by the 4th PCs. Thus
there seems there is some improvement.

\hypertarget{plot-of-pc1-vs-pc2-1}{%
\subsubsection{Plot of PC1 vs PC2}\label{plot-of-pc1-vs-pc2-1}}

\begin{Shaded}
\begin{Highlighting}[]
\NormalTok{PcaDf2 <-}\StringTok{ }\KeywordTok{as.data.frame}\NormalTok{(PCAData2}\OperatorTok{$}\NormalTok{x)}
\KeywordTok{ggplot}\NormalTok{(PcaDf2, }\KeywordTok{aes}\NormalTok{(}\DataTypeTok{x=}\NormalTok{PC1, }\DataTypeTok{y=}\NormalTok{PC2, }\DataTypeTok{col=}\NormalTok{data}\OperatorTok{$}\NormalTok{class)) }\OperatorTok{+}\StringTok{ }\KeywordTok{geom_point}\NormalTok{(}\DataTypeTok{alpha=}\FloatTok{0.5}\NormalTok{)}
\end{Highlighting}
\end{Shaded}

\includegraphics{NETCapstone_files/figure-latex/unnamed-chunk-19-1.pdf}

From the plot above it can be determined that the first two components
are still somewhat separated into two classes. This is caused by the
fact that the variance explained by these components is not large. There
is some change in the density plots, the density plots have narrowed for
the components and thus this can be see as some improvement.

\hypertarget{plot-of-densitys-1}{%
\subsubsection{Plot of densitys}\label{plot-of-densitys-1}}

\begin{Shaded}
\begin{Highlighting}[]
\NormalTok{pc12 <-}\StringTok{ }\KeywordTok{ggplot}\NormalTok{(PcaDf2, }\KeywordTok{aes}\NormalTok{(}\DataTypeTok{x=}\NormalTok{PC1, }\DataTypeTok{fill=}\NormalTok{data}\OperatorTok{$}\NormalTok{class)) }\OperatorTok{+}\StringTok{ }\KeywordTok{geom_density}\NormalTok{(}\DataTypeTok{alpha=}\FloatTok{0.25}\NormalTok{)  }
\NormalTok{pc22 <-}\StringTok{ }\KeywordTok{ggplot}\NormalTok{(PcaDf2, }\KeywordTok{aes}\NormalTok{(}\DataTypeTok{x=}\NormalTok{PC2, }\DataTypeTok{fill=}\NormalTok{data}\OperatorTok{$}\NormalTok{class)) }\OperatorTok{+}\StringTok{ }\KeywordTok{geom_density}\NormalTok{(}\DataTypeTok{alpha=}\FloatTok{0.25}\NormalTok{)  }
\KeywordTok{grid.arrange}\NormalTok{(pc12, pc22, }\DataTypeTok{ncol=}\DecValTok{2}\NormalTok{)}
\end{Highlighting}
\end{Shaded}

\includegraphics{NETCapstone_files/figure-latex/unnamed-chunk-20-1.pdf}

\hypertarget{linear-discriminant-analysis-lda}{%
\section{Linear Discriminant Analysis
{[}LDA{]}}\label{linear-discriminant-analysis-lda}}

Other than PCA, there is LDA. LDA takes in consideration the different
classes and could possibly get better results.

The particularity of LDA is that it models the distribution of
predictors separately in each of the response classes, and then it uses
Bayes' theorem to estimate the probability. It is important to note that
LDA assumes a normal distribution for each class, a class-specific mean,
and a common variance. (Peixeiro, n.d.)

\hypertarget{lda-with-original-data}{%
\subsection{LDA with original data}\label{lda-with-original-data}}

\begin{Shaded}
\begin{Highlighting}[]
\NormalTok{LdaData <-}\StringTok{ }\NormalTok{MASS}\OperatorTok{::}\KeywordTok{lda}\NormalTok{(class}\OperatorTok{~}\NormalTok{., }\DataTypeTok{data =}\NormalTok{ data, }\DataTypeTok{center =} \OtherTok{TRUE}\NormalTok{, }\DataTypeTok{scale =} \OtherTok{TRUE}\NormalTok{) }
\NormalTok{LdaData}
\end{Highlighting}
\end{Shaded}

\begin{verbatim}
## Call:
## lda(class ~ ., data = data, center = TRUE, scale = TRUE)
## 
## Prior probabilities of groups:
##  Abnormal    Normal 
## 0.6774194 0.3225806 
## 
## Group means:
##          pelvic_incidence pelvic_tilt.numeric lumbar_lordosis_angle
## Abnormal         64.69256            19.79111              55.92537
## Normal           51.68524            12.82141              43.54260
##          sacral_slope pelvic_radius degree_spondylolisthesis
## Abnormal     44.90145      115.0777                37.777705
## Normal       38.86383      123.8908                 2.186572
## 
## Coefficients of linear discriminants:
##                                   LD1
## pelvic_incidence          0.007884392
## pelvic_tilt.numeric      -0.032981545
## lumbar_lordosis_angle    -0.015839324
## sacral_slope              0.029051984
## pelvic_radius             0.058779826
## degree_spondylolisthesis -0.024406328
\end{verbatim}

\begin{Shaded}
\begin{Highlighting}[]
\CommentTok{#Data frame of the LDA for visualization purposes}
\NormalTok{ldaDataPredict <-}\StringTok{ }\KeywordTok{predict}\NormalTok{(LdaData, data)}\OperatorTok{$}\NormalTok{x }\OperatorTok\StringTok{ }\KeywordTok{as.data.frame}\NormalTok{() }\OperatorTok\StringTok{ }\KeywordTok{cbind}\NormalTok{(}\DataTypeTok{class=}\NormalTok{data}\OperatorTok{$}\NormalTok{class)}
\end{Highlighting}
\end{Shaded}

\hypertarget{plot-density-of-ld1}{%
\subsubsection{Plot density of LD1}\label{plot-density-of-ld1}}

\begin{Shaded}
\begin{Highlighting}[]
\KeywordTok{ggplot}\NormalTok{(ldaDataPredict, }\KeywordTok{aes}\NormalTok{(}\DataTypeTok{x=}\NormalTok{LD1, }\DataTypeTok{fill=}\NormalTok{class)) }\OperatorTok{+}\StringTok{ }\KeywordTok{geom_density}\NormalTok{(}\DataTypeTok{alpha=}\FloatTok{0.5}\NormalTok{)}
\end{Highlighting}
\end{Shaded}

\includegraphics{NETCapstone_files/figure-latex/unnamed-chunk-22-1.pdf}

\hypertarget{lda-on-cleaned-dataset}{%
\subsection{LDA on Cleaned Dataset}\label{lda-on-cleaned-dataset}}

\begin{Shaded}
\begin{Highlighting}[]
\NormalTok{LdaData2 <-}\StringTok{ }\NormalTok{MASS}\OperatorTok{::}\KeywordTok{lda}\NormalTok{(class}\OperatorTok{~}\NormalTok{., }\DataTypeTok{data =}\NormalTok{ data2, }\DataTypeTok{center =} \OtherTok{TRUE}\NormalTok{, }\DataTypeTok{scale =} \OtherTok{TRUE}\NormalTok{) }
\NormalTok{LdaData2}
\end{Highlighting}
\end{Shaded}

\begin{verbatim}
## Call:
## lda(class ~ ., data = data2, center = TRUE, scale = TRUE)
## 
## Prior probabilities of groups:
##  Abnormal    Normal 
## 0.6774194 0.3225806 
## 
## Group means:
##          pelvic_tilt.numeric lumbar_lordosis_angle sacral_slope pelvic_radius
## Abnormal            19.79111              55.92537     44.90145      115.0777
## Normal              12.82141              43.54260     38.86383      123.8908
##          degree_spondylolisthesis
## Abnormal                37.777705
## Normal                   2.186572
## 
## Coefficients of linear discriminants:
##                                  LD1
## pelvic_tilt.numeric      -0.02509715
## lumbar_lordosis_angle    -0.01583932
## sacral_slope              0.03693638
## pelvic_radius             0.05877983
## degree_spondylolisthesis -0.02440633
\end{verbatim}

\begin{Shaded}
\begin{Highlighting}[]
\CommentTok{#Data frame of the LDA for visualization purposes}
\NormalTok{ldaDataPredict2 <-}\StringTok{ }\KeywordTok{predict}\NormalTok{(LdaData2, data2)}\OperatorTok{$}\NormalTok{x }\OperatorTok\StringTok{ }\KeywordTok{as.data.frame}\NormalTok{() }\OperatorTok\StringTok{ }\KeywordTok{cbind}\NormalTok{(}\DataTypeTok{class=}\NormalTok{data2}\OperatorTok{$}\NormalTok{class)}
\end{Highlighting}
\end{Shaded}

\hypertarget{plot-density-of-ld1-1}{%
\subsubsection{Plot density of LD1}\label{plot-density-of-ld1-1}}

\begin{Shaded}
\begin{Highlighting}[]
\KeywordTok{ggplot}\NormalTok{(ldaDataPredict2, }\KeywordTok{aes}\NormalTok{(}\DataTypeTok{x=}\NormalTok{LD1, }\DataTypeTok{fill=}\NormalTok{class)) }\OperatorTok{+}\StringTok{ }\KeywordTok{geom_density}\NormalTok{(}\DataTypeTok{alpha=}\FloatTok{0.5}\NormalTok{)}
\end{Highlighting}
\end{Shaded}

\includegraphics{NETCapstone_files/figure-latex/unnamed-chunk-24-1.pdf}
There seems to be very little difference in the LDA outputs and plot
when using the original and cleaned Datasets. The only changes that can
be noted is the changes in the Coefficients of linear discriminates.

There is a clearer difference between the Normal and Abnormal classes in
the density plots when using LDA

\newpage

\hypertarget{methods}{%
\section{Methods}\label{methods}}

\hypertarget{original-dataset}{%
\subsection{Original dataset}\label{original-dataset}}

\hypertarget{creating-train-and-trainingvalidation-datasets}{%
\subsubsection{Creating train and training/validation
datasets}\label{creating-train-and-trainingvalidation-datasets}}

\begin{Shaded}
\begin{Highlighting}[]
\KeywordTok{set.seed}\NormalTok{(}\DecValTok{1}\NormalTok{, }\DataTypeTok{sample.kind=}\StringTok{"Rounding"}\NormalTok{) }\CommentTok{#if using R 3.5 or earlier, use `set.seed(1)}
\CommentTok{#Train (80% of data)}
\NormalTok{datSamplingIndex <-}\StringTok{ }\KeywordTok{createDataPartition}\NormalTok{(data}\OperatorTok{$}\NormalTok{class, }\DataTypeTok{times=}\DecValTok{1}\NormalTok{, }\DataTypeTok{p=}\FloatTok{0.8}\NormalTok{, }\DataTypeTok{list =} \OtherTok{FALSE}\NormalTok{)}
\NormalTok{trainData <-}\StringTok{ }\NormalTok{data[datSamplingIndex, ]}
\CommentTok{#Test (20%)}
\NormalTok{testData <-}\StringTok{ }\NormalTok{data[}\OperatorTok{-}\NormalTok{datSamplingIndex, ]}
\end{Highlighting}
\end{Shaded}

\hypertarget{control-the-data}{%
\subsubsection{Control the data}\label{control-the-data}}

\begin{Shaded}
\begin{Highlighting}[]
\NormalTok{fitControl <-}\StringTok{ }\KeywordTok{trainControl}\NormalTok{(}\DataTypeTok{method=}\StringTok{"cv"}\NormalTok{,    }
                           \DataTypeTok{number =} \DecValTok{20}\NormalTok{,    }
                           \DataTypeTok{classProbs =} \OtherTok{TRUE}\NormalTok{,}
                           \DataTypeTok{summaryFunction =}\NormalTok{ twoClassSummary)}
\end{Highlighting}
\end{Shaded}

\hypertarget{cleaned-dataset}{%
\subsection{Cleaned Dataset}\label{cleaned-dataset}}

\hypertarget{creating-train-and-trainingvalidation-datasets-1}{%
\subsubsection{Creating train and training/validation
datasets}\label{creating-train-and-trainingvalidation-datasets-1}}

\begin{Shaded}
\begin{Highlighting}[]
\CommentTok{#Train (80% of data)}
\NormalTok{data3 <-}\StringTok{ }\KeywordTok{cbind}\NormalTok{ (}\DataTypeTok{class=}\NormalTok{data}\OperatorTok{$}\NormalTok{class, data2)}
\NormalTok{trainData2 <-}\StringTok{ }\NormalTok{data3[datSamplingIndex, ]}
\CommentTok{#Test (20%)}
\NormalTok{testData2 <-}\StringTok{ }\NormalTok{data3[}\OperatorTok{-}\NormalTok{datSamplingIndex, ]}
\end{Highlighting}
\end{Shaded}

\hypertarget{control-the-data-1}{%
\subsubsection{Control the data}\label{control-the-data-1}}

\begin{Shaded}
\begin{Highlighting}[]
\NormalTok{fitControl2 <-}\StringTok{ }\KeywordTok{trainControl}\NormalTok{(}\DataTypeTok{method=}\StringTok{"cv"}\NormalTok{,    }
                           \DataTypeTok{number =} \DecValTok{20}\NormalTok{,    }
                           \DataTypeTok{classProbs =} \OtherTok{TRUE}\NormalTok{,}
                           \DataTypeTok{summaryFunction =}\NormalTok{ twoClassSummary)}
\end{Highlighting}
\end{Shaded}

\hypertarget{naive-bayes-model}{%
\subsection{Naive Bayes Model}\label{naive-bayes-model}}

The Naive Bayesian classifier is based on Bayes' theorem with the
independence assumptions between predictors. A Naive Bayesian model is
simple to build, as there are no complicated iterative parameter
estimation. Bayes theorem provides a way of calculating the posterior
probability, P(c\textbar x), from P(c), P(x), and P(x\textbar c). Naive
Bayes classifier assumes that the effect of a predictor (x) value on a
given class (c) is independent of the values of other predictors. This
assumption is called class conditional independence.

Although the Naive Bayesian classifier is simplistic it often works
well. Thus it is widely use as it often outperforms more sophisticated
classification methods. (``Naive Bayesian,'' n.d.)

\hypertarget{original-dataset-1}{%
\subsubsection{Original Dataset}\label{original-dataset-1}}

\begin{Shaded}
\begin{Highlighting}[]
\NormalTok{NaiveBayesModel <-}\StringTok{ }\KeywordTok{train}\NormalTok{(class}\OperatorTok{~}\NormalTok{.,}
\NormalTok{                      trainData,}
                      \DataTypeTok{method=}\StringTok{"nb"}\NormalTok{,}
                      \DataTypeTok{metric=}\StringTok{"ROC"}\NormalTok{,}
                      \DataTypeTok{preProcess=}\KeywordTok{c}\NormalTok{(}\StringTok{'center'}\NormalTok{, }\StringTok{'scale'}\NormalTok{), }\CommentTok{#in order to normalize the data}
                      \DataTypeTok{trace=}\OtherTok{FALSE}\NormalTok{, }
                      \DataTypeTok{importance =} \OtherTok{TRUE}\NormalTok{,}
                      \DataTypeTok{trControl=}\NormalTok{fitControl)}

\NormalTok{NBPrediction <-}\StringTok{ }\KeywordTok{predict}\NormalTok{(NaiveBayesModel, testData)}
\NormalTok{NBConfMatrix <-}\StringTok{ }\KeywordTok{confusionMatrix}\NormalTok{(NBPrediction, }\KeywordTok{as.factor}\NormalTok{(testData}\OperatorTok{$}\NormalTok{class), }\DataTypeTok{positive =} \StringTok{"Abnormal"}\NormalTok{)}
\NormalTok{NBConfMatrix}
\end{Highlighting}
\end{Shaded}

\begin{verbatim}
## Confusion Matrix and Statistics
## 
##           Reference
## Prediction Abnormal Normal
##   Abnormal       25      2
##   Normal         17     18
##                                           
##                Accuracy : 0.6935          
##                  95% CI : (0.5635, 0.8044)
##     No Information Rate : 0.6774          
##     P-Value [Acc > NIR] : 0.452354        
##                                           
##                   Kappa : 0.4139          
##                                           
##  Mcnemar's Test P-Value : 0.001319        
##                                           
##             Sensitivity : 0.5952          
##             Specificity : 0.9000          
##          Pos Pred Value : 0.9259          
##          Neg Pred Value : 0.5143          
##              Prevalence : 0.6774          
##          Detection Rate : 0.4032          
##    Detection Prevalence : 0.4355          
##       Balanced Accuracy : 0.7476          
##                                           
##        'Positive' Class : Abnormal        
## 
\end{verbatim}

\hypertarget{cleaned-dataset-1}{%
\subsubsection{Cleaned Dataset}\label{cleaned-dataset-1}}

\begin{Shaded}
\begin{Highlighting}[]
\NormalTok{NaiveBayesModel2 <-}\StringTok{ }\KeywordTok{train}\NormalTok{(class}\OperatorTok{~}\NormalTok{.,}
\NormalTok{                      trainData2,}
                      \DataTypeTok{method=}\StringTok{"nb"}\NormalTok{,}
                      \DataTypeTok{metric=}\StringTok{"ROC"}\NormalTok{,}
                      \DataTypeTok{preProcess=}\KeywordTok{c}\NormalTok{(}\StringTok{'center'}\NormalTok{, }\StringTok{'scale'}\NormalTok{), }\CommentTok{#in order to normalize the data}
                      \DataTypeTok{trace=}\OtherTok{FALSE}\NormalTok{, }
                      \DataTypeTok{importance =} \OtherTok{TRUE}\NormalTok{,}
                      \DataTypeTok{trControl=}\NormalTok{fitControl2)}

\NormalTok{NBPrediction2 <-}\StringTok{ }\KeywordTok{predict}\NormalTok{(NaiveBayesModel2, testData2)}
\NormalTok{NBConfMatrix2 <-}\StringTok{ }\KeywordTok{confusionMatrix}\NormalTok{(NBPrediction2, }\KeywordTok{as.factor}\NormalTok{(testData2}\OperatorTok{$}\NormalTok{class), }\DataTypeTok{positive =} \StringTok{"Abnormal"}\NormalTok{)}
\NormalTok{NBConfMatrix2}
\end{Highlighting}
\end{Shaded}

\begin{verbatim}
## Confusion Matrix and Statistics
## 
##           Reference
## Prediction Abnormal Normal
##   Abnormal       27      1
##   Normal         15     19
##                                          
##                Accuracy : 0.7419         
##                  95% CI : (0.615, 0.8447)
##     No Information Rate : 0.6774         
##     P-Value [Acc > NIR] : 0.171207       
##                                          
##                   Kappa : 0.501          
##                                          
##  Mcnemar's Test P-Value : 0.001154       
##                                          
##             Sensitivity : 0.6429         
##             Specificity : 0.9500         
##          Pos Pred Value : 0.9643         
##          Neg Pred Value : 0.5588         
##              Prevalence : 0.6774         
##          Detection Rate : 0.4355         
##    Detection Prevalence : 0.4516         
##       Balanced Accuracy : 0.7964         
##                                          
##        'Positive' Class : Abnormal       
## 
\end{verbatim}

There is a positive effect on the Naive Bayes Model from the removal of
the highly correlated variable.

The accuracy of the Cleaned Dataset model can be noted being 0.7419
where as the Original Dataset model had and accuracy of 0.6935. It seems
that the Naive Bayes Model is performing fairly well

In a later discussion other metrics will be discussed, such as:

\begin{itemize}
\tightlist
\item
  Sensitivity (recall) represent the true positive rate: the proportions
  of actual positives correctly identified.
\item
  Specificity is the true negative rate: the proportion of actual
  negatives correctly identified.
\item
  Accuracy is the general score of the classifier model performance as
  it is the ratio of how many samples are correctly classified to all
  samples.
\item
  F1 score: the harmonic mean of precision and sensitivity.
\item
  Accuracy and F1 score would be used to compare the result with the
  benchmark model.
\item
  Precision: the number of correct positive results divided by the
  number of all positive results returned by the classifier.
\end{itemize}

\hypertarget{random-forest-rf}{%
\subsection{Random Forest {[}RF{]}}\label{random-forest-rf}}

The RF is one of the most powerful machine learning algorithms
available. RF is a supervised machine learning algorithm that can be
used for both classification and regression tasks. The algorithm
addresses the shortcomings of decision trees by using a clever tick. Its
goal is to improve prediction performance and reduce instability by
averaging multiple decision trees. RF is made from a group of individual
decision trees, this technique is called Ensemble Learning. A large
group of uncorrelated decision trees can produce more accurate and
stable results than any of individual decision trees.

Training a RF for a classification task, is actually training a group of
decision trees. Then by obtaining all the predictions of each individual
trees and can use these predictions to predict the class that gets the
most votes. Although some individual trees produce wrong predictions,
many can produce accurate predictions. As a group, they can move towards
accurate predictions. (Pramoditha, n.d.)

\hypertarget{original-dataset-2}{%
\subsubsection{Original Dataset}\label{original-dataset-2}}

\begin{Shaded}
\begin{Highlighting}[]
\NormalTok{RandomforestModel <-}\StringTok{ }\KeywordTok{train}\NormalTok{(class}\OperatorTok{~}\NormalTok{.,}
\NormalTok{                            trainData,}
                            \DataTypeTok{method=}\StringTok{"rf"}\NormalTok{,  }
                            \DataTypeTok{metric=}\StringTok{"ROC"}\NormalTok{,}
                            \CommentTok{#tuneLength=10,}
                            \CommentTok{#tuneGrid = expand.grid(mtry = c(2, 3, 6)),}
                            \DataTypeTok{preProcess =} \KeywordTok{c}\NormalTok{(}\StringTok{'center'}\NormalTok{, }\StringTok{'scale'}\NormalTok{),}
                            \DataTypeTok{trControl=}\NormalTok{fitControl)}

\NormalTok{RFPrediction <-}\StringTok{ }\KeywordTok{predict}\NormalTok{(RandomforestModel, testData)}
\CommentTok{#Check results}
\NormalTok{RFConfMatrix <-}\StringTok{ }\KeywordTok{confusionMatrix}\NormalTok{(RFPrediction, }\KeywordTok{as.factor}\NormalTok{(testData}\OperatorTok{$}\NormalTok{class), }\DataTypeTok{positive =} \StringTok{"Abnormal"}\NormalTok{)}
\NormalTok{RFConfMatrix}
\end{Highlighting}
\end{Shaded}

\begin{verbatim}
## Confusion Matrix and Statistics
## 
##           Reference
## Prediction Abnormal Normal
##   Abnormal       35      5
##   Normal          7     15
##                                           
##                Accuracy : 0.8065          
##                  95% CI : (0.6863, 0.8958)
##     No Information Rate : 0.6774          
##     P-Value [Acc > NIR] : 0.01765         
##                                           
##                   Kappa : 0.5684          
##                                           
##  Mcnemar's Test P-Value : 0.77283         
##                                           
##             Sensitivity : 0.8333          
##             Specificity : 0.7500          
##          Pos Pred Value : 0.8750          
##          Neg Pred Value : 0.6818          
##              Prevalence : 0.6774          
##          Detection Rate : 0.5645          
##    Detection Prevalence : 0.6452          
##       Balanced Accuracy : 0.7917          
##                                           
##        'Positive' Class : Abnormal        
## 
\end{verbatim}

\hypertarget{plot-of-variable-importance}{%
\paragraph{Plot of Variable
Importance}\label{plot-of-variable-importance}}

\begin{Shaded}
\begin{Highlighting}[]
\KeywordTok{varImp}\NormalTok{(RandomforestModel)}
\end{Highlighting}
\end{Shaded}

\begin{verbatim}
## rf variable importance
## 
##                          Overall
## degree_spondylolisthesis 100.000
## pelvic_radius             22.152
## pelvic_incidence           6.497
## pelvic_tilt.numeric        2.580
## lumbar_lordosis_angle      1.760
## sacral_slope               0.000
\end{verbatim}

\begin{Shaded}
\begin{Highlighting}[]
\KeywordTok{plot}\NormalTok{(}\KeywordTok{varImp}\NormalTok{(RandomforestModel), }\DataTypeTok{main=}\StringTok{"Ranking Of Importance"}\NormalTok{)}
\end{Highlighting}
\end{Shaded}

\includegraphics{NETCapstone_files/figure-latex/unnamed-chunk-33-1.pdf}

\hypertarget{cleaned-dataset-2}{%
\subsubsection{Cleaned Dataset}\label{cleaned-dataset-2}}

\begin{Shaded}
\begin{Highlighting}[]
\NormalTok{RandomforestModel2 <-}\StringTok{ }\KeywordTok{train}\NormalTok{(class}\OperatorTok{~}\NormalTok{.,}
\NormalTok{                            trainData2,}
                            \DataTypeTok{method=}\StringTok{"rf"}\NormalTok{,}
                            \DataTypeTok{metric=}\StringTok{"ROC"}\NormalTok{,}
                            \DataTypeTok{preProcess =} \KeywordTok{c}\NormalTok{(}\StringTok{'center'}\NormalTok{, }\StringTok{'scale'}\NormalTok{),}
                            \DataTypeTok{trControl=}\NormalTok{fitControl2)}

\NormalTok{RFPrediction2 <-}\StringTok{ }\KeywordTok{predict}\NormalTok{(RandomforestModel2, testData2)}
\CommentTok{#Check results}
\NormalTok{RFConfMatrix2 <-}\StringTok{ }\KeywordTok{confusionMatrix}\NormalTok{(RFPrediction2, }\KeywordTok{as.factor}\NormalTok{(testData2}\OperatorTok{$}\NormalTok{class), }\DataTypeTok{positive =} \StringTok{"Abnormal"}\NormalTok{)}
\NormalTok{RFConfMatrix2}
\end{Highlighting}
\end{Shaded}

\begin{verbatim}
## Confusion Matrix and Statistics
## 
##           Reference
## Prediction Abnormal Normal
##   Abnormal       35      6
##   Normal          7     14
##                                           
##                Accuracy : 0.7903          
##                  95% CI : (0.6682, 0.8834)
##     No Information Rate : 0.6774          
##     P-Value [Acc > NIR] : 0.03518         
##                                           
##                   Kappa : 0.5264          
##                                           
##  Mcnemar's Test P-Value : 1.00000         
##                                           
##             Sensitivity : 0.8333          
##             Specificity : 0.7000          
##          Pos Pred Value : 0.8537          
##          Neg Pred Value : 0.6667          
##              Prevalence : 0.6774          
##          Detection Rate : 0.5645          
##    Detection Prevalence : 0.6613          
##       Balanced Accuracy : 0.7667          
##                                           
##        'Positive' Class : Abnormal        
## 
\end{verbatim}

\hypertarget{plot-of-variable-importance-1}{%
\paragraph{Plot of Variable
Importance}\label{plot-of-variable-importance-1}}

\begin{Shaded}
\begin{Highlighting}[]
\KeywordTok{varImp}\NormalTok{(RandomforestModel2)}
\end{Highlighting}
\end{Shaded}

\begin{verbatim}
## rf variable importance
## 
##                          Overall
## degree_spondylolisthesis 100.000
## pelvic_radius             13.310
## lumbar_lordosis_angle      5.252
## pelvic_tilt.numeric        4.658
## sacral_slope               0.000
\end{verbatim}

\begin{Shaded}
\begin{Highlighting}[]
\KeywordTok{plot}\NormalTok{(}\KeywordTok{varImp}\NormalTok{(RandomforestModel2), }\DataTypeTok{main=}\StringTok{"Ranking Of Importance"}\NormalTok{)}
\end{Highlighting}
\end{Shaded}

\includegraphics{NETCapstone_files/figure-latex/unnamed-chunk-36-1.pdf}

The removal of the highly correlated variable from the original dataset
had negitive impact on the Random Forest Model. This can be seen from
the decrease in the Accuracy and Specificity.

\hypertarget{logistic-regression-model-logreg}{%
\subsection{Logistic Regression Model
{[}LogReg{]}}\label{logistic-regression-model-logreg}}

LogReg is mainly used for binary classification. A binary logistic model
is used to estimate the probability of a binary response based on one or
more predictor (or independent) variables (features).

\hypertarget{original-dataset-3}{%
\subsubsection{Original Dataset}\label{original-dataset-3}}

\begin{Shaded}
\begin{Highlighting}[]
\NormalTok{LogRegModel<-}\StringTok{ }\KeywordTok{train}\NormalTok{(class}\OperatorTok{~}\NormalTok{., }\DataTypeTok{data =}\NormalTok{ trainData, }
                     \DataTypeTok{method =} \StringTok{"glm"}\NormalTok{,}
                     \DataTypeTok{metric =} \StringTok{"ROC"}\NormalTok{,}
                     \DataTypeTok{preProcess =} \KeywordTok{c}\NormalTok{(}\StringTok{"scale"}\NormalTok{, }\StringTok{"center"}\NormalTok{),  }\CommentTok{# in order to normalize the data}
                     \DataTypeTok{trControl=}\NormalTok{ fitControl)}

\NormalTok{LogRegPred <-}\StringTok{ }\KeywordTok{predict}\NormalTok{(LogRegModel, testData)}
\CommentTok{# Check results}
\NormalTok{LogRegConfMatrix <-}\StringTok{ }\KeywordTok{confusionMatrix}\NormalTok{(LogRegPred, }\KeywordTok{as.factor}\NormalTok{(testData}\OperatorTok{$}\NormalTok{class), }\DataTypeTok{positive =} \StringTok{"Abnormal"}\NormalTok{)}
\NormalTok{LogRegConfMatrix}
\end{Highlighting}
\end{Shaded}

\begin{verbatim}
## Confusion Matrix and Statistics
## 
##           Reference
## Prediction Abnormal Normal
##   Abnormal       36      5
##   Normal          6     15
##                                          
##                Accuracy : 0.8226         
##                  95% CI : (0.7047, 0.908)
##     No Information Rate : 0.6774         
##     P-Value [Acc > NIR] : 0.008074       
##                                          
##                   Kappa : 0.5993         
##                                          
##  Mcnemar's Test P-Value : 1.000000       
##                                          
##             Sensitivity : 0.8571         
##             Specificity : 0.7500         
##          Pos Pred Value : 0.8780         
##          Neg Pred Value : 0.7143         
##              Prevalence : 0.6774         
##          Detection Rate : 0.5806         
##    Detection Prevalence : 0.6613         
##       Balanced Accuracy : 0.8036         
##                                          
##        'Positive' Class : Abnormal       
## 
\end{verbatim}

\hypertarget{plot-of-variable-importance-2}{%
\paragraph{Plot of Variable
Importance}\label{plot-of-variable-importance-2}}

\begin{Shaded}
\begin{Highlighting}[]
\KeywordTok{varImp}\NormalTok{(LogRegModel)}
\end{Highlighting}
\end{Shaded}

\begin{verbatim}
## glm variable importance
## 
##                            Overall
## degree_spondylolisthesis 1.000e+02
## pelvic_radius            6.079e+01
## lumbar_lordosis_angle    2.707e+00
## pelvic_tilt.numeric      6.557e-08
## pelvic_incidence         3.134e-08
## sacral_slope             0.000e+00
\end{verbatim}

\begin{Shaded}
\begin{Highlighting}[]
\KeywordTok{plot}\NormalTok{(}\KeywordTok{varImp}\NormalTok{(LogRegModel), }\DataTypeTok{main=}\StringTok{"Ranking Of Importance"}\NormalTok{)}
\end{Highlighting}
\end{Shaded}

\includegraphics{NETCapstone_files/figure-latex/unnamed-chunk-39-1.pdf}

\hypertarget{cleaned-dataset-3}{%
\subsubsection{Cleaned Dataset}\label{cleaned-dataset-3}}

\begin{Shaded}
\begin{Highlighting}[]
\NormalTok{LogRegModel2<-}\StringTok{ }\KeywordTok{train}\NormalTok{(class}\OperatorTok{~}\NormalTok{., }\DataTypeTok{data =}\NormalTok{ trainData2, }
                     \DataTypeTok{method =} \StringTok{"glm"}\NormalTok{,}
                     \DataTypeTok{metric =} \StringTok{"ROC"}\NormalTok{,}
                     \DataTypeTok{preProcess =} \KeywordTok{c}\NormalTok{(}\StringTok{"scale"}\NormalTok{, }\StringTok{"center"}\NormalTok{),  }\CommentTok{# in order to normalize the data}
                     \DataTypeTok{trControl=}\NormalTok{ fitControl2)}

\NormalTok{LogRegPred2 <-}\StringTok{ }\KeywordTok{predict}\NormalTok{(LogRegModel2, testData2)}
\CommentTok{# Check results}
\NormalTok{LogRegConfMatrix2 <-}\StringTok{ }\KeywordTok{confusionMatrix}\NormalTok{(LogRegPred2, }\KeywordTok{as.factor}\NormalTok{(testData2}\OperatorTok{$}\NormalTok{class), }\DataTypeTok{positive =} \StringTok{"Abnormal"}\NormalTok{)}
\NormalTok{LogRegConfMatrix2}
\end{Highlighting}
\end{Shaded}

\begin{verbatim}
## Confusion Matrix and Statistics
## 
##           Reference
## Prediction Abnormal Normal
##   Abnormal       36      5
##   Normal          6     15
##                                          
##                Accuracy : 0.8226         
##                  95% CI : (0.7047, 0.908)
##     No Information Rate : 0.6774         
##     P-Value [Acc > NIR] : 0.008074       
##                                          
##                   Kappa : 0.5993         
##                                          
##  Mcnemar's Test P-Value : 1.000000       
##                                          
##             Sensitivity : 0.8571         
##             Specificity : 0.7500         
##          Pos Pred Value : 0.8780         
##          Neg Pred Value : 0.7143         
##              Prevalence : 0.6774         
##          Detection Rate : 0.5806         
##    Detection Prevalence : 0.6613         
##       Balanced Accuracy : 0.8036         
##                                          
##        'Positive' Class : Abnormal       
## 
\end{verbatim}

\hypertarget{plot-of-variable-importance-3}{%
\paragraph{Plot of Variable
Importance}\label{plot-of-variable-importance-3}}

\begin{Shaded}
\begin{Highlighting}[]
\KeywordTok{varImp}\NormalTok{(LogRegModel2)}
\end{Highlighting}
\end{Shaded}

\begin{verbatim}
## glm variable importance
## 
##                          Overall
## degree_spondylolisthesis  100.00
## pelvic_radius              59.69
## pelvic_tilt.numeric        23.80
## sacral_slope               21.78
## lumbar_lordosis_angle       0.00
\end{verbatim}

\begin{Shaded}
\begin{Highlighting}[]
\KeywordTok{plot}\NormalTok{(}\KeywordTok{varImp}\NormalTok{(LogRegModel2), }\DataTypeTok{main=}\StringTok{"Ranking Of Importance"}\NormalTok{)}
\end{Highlighting}
\end{Shaded}

\includegraphics{NETCapstone_files/figure-latex/unnamed-chunk-42-1.pdf}

There is no impact on the LogReg model with the removal of the highly
correlated variable.

\hypertarget{k-nearest-neighbor-knn-model}{%
\subsection{K Nearest Neighbor (KNN)
Model}\label{k-nearest-neighbor-knn-model}}

KNN is a supervised learning algorithms that is commonly used in data
mining and machine learning. It is a classifier algorithm, learning is
based ``how similar'' is a data from one another. KNN is a simple
algorithm that stores all available cases and classifies new cases based
on a similarity measure. The KNN algorithm assumes that similar things
exist in close proximity (Harrison, n.d.).

\hypertarget{original-dataset-4}{%
\subsubsection{Original Dataset}\label{original-dataset-4}}

\begin{Shaded}
\begin{Highlighting}[]
\NormalTok{KNNModel <-}\StringTok{ }\KeywordTok{train}\NormalTok{(class}\OperatorTok{~}\NormalTok{.,}
\NormalTok{                   trainData,}
                   \DataTypeTok{method=}\StringTok{"knn"}\NormalTok{,}
                   \DataTypeTok{metric=}\StringTok{"ROC"}\NormalTok{,}
                   \DataTypeTok{preProcess =} \KeywordTok{c}\NormalTok{(}\StringTok{'center'}\NormalTok{, }\StringTok{'scale'}\NormalTok{),}
                   \DataTypeTok{tuneLength=}\DecValTok{10}\NormalTok{, }
                   \DataTypeTok{trControl=}\NormalTok{fitControl)}

\NormalTok{KNNPred <-}\StringTok{ }\KeywordTok{predict}\NormalTok{(KNNModel, testData)}
\NormalTok{KNNConfMatrix <-}\StringTok{ }\KeywordTok{confusionMatrix}\NormalTok{(KNNPred, }\KeywordTok{as.factor}\NormalTok{(testData}\OperatorTok{$}\NormalTok{class), }\DataTypeTok{positive =} \StringTok{"Abnormal"}\NormalTok{)}
\NormalTok{KNNConfMatrix}
\end{Highlighting}
\end{Shaded}

\begin{verbatim}
## Confusion Matrix and Statistics
## 
##           Reference
## Prediction Abnormal Normal
##   Abnormal       32      6
##   Normal         10     14
##                                          
##                Accuracy : 0.7419         
##                  95% CI : (0.615, 0.8447)
##     No Information Rate : 0.6774         
##     P-Value [Acc > NIR] : 0.1712         
##                                          
##                   Kappa : 0.4389         
##                                          
##  Mcnemar's Test P-Value : 0.4533         
##                                          
##             Sensitivity : 0.7619         
##             Specificity : 0.7000         
##          Pos Pred Value : 0.8421         
##          Neg Pred Value : 0.5833         
##              Prevalence : 0.6774         
##          Detection Rate : 0.5161         
##    Detection Prevalence : 0.6129         
##       Balanced Accuracy : 0.7310         
##                                          
##        'Positive' Class : Abnormal       
## 
\end{verbatim}

\hypertarget{plot-of-roc-vs-number-of-neighbors}{%
\paragraph{Plot of ROC vs Number of
Neighbors}\label{plot-of-roc-vs-number-of-neighbors}}

\begin{Shaded}
\begin{Highlighting}[]
\KeywordTok{print}\NormalTok{(}\KeywordTok{ggplot}\NormalTok{(KNNModel))}
\end{Highlighting}
\end{Shaded}

\begin{center}\includegraphics{NETCapstone_files/figure-latex/unnamed-chunk-44-1} \end{center}

\hypertarget{cleaned-dataset-4}{%
\subsubsection{Cleaned Dataset}\label{cleaned-dataset-4}}

\begin{Shaded}
\begin{Highlighting}[]
\NormalTok{KNNModel2 <-}\StringTok{ }\KeywordTok{train}\NormalTok{(class}\OperatorTok{~}\NormalTok{.,}
\NormalTok{                   trainData2,}
                   \DataTypeTok{method=}\StringTok{"knn"}\NormalTok{,}
                   \DataTypeTok{metric=}\StringTok{"ROC"}\NormalTok{,}
                   \DataTypeTok{preProcess =} \KeywordTok{c}\NormalTok{(}\StringTok{'center'}\NormalTok{, }\StringTok{'scale'}\NormalTok{),}
                   \DataTypeTok{tuneLength=}\DecValTok{10}\NormalTok{, }
                   \DataTypeTok{trControl=}\NormalTok{fitControl2)}

\NormalTok{KNNPred2 <-}\StringTok{ }\KeywordTok{predict}\NormalTok{(KNNModel2, testData2)}
\NormalTok{KNNConfMatrix2 <-}\StringTok{ }\KeywordTok{confusionMatrix}\NormalTok{(KNNPred2, }\KeywordTok{as.factor}\NormalTok{(testData2}\OperatorTok{$}\NormalTok{class), }\DataTypeTok{positive =} \StringTok{"Abnormal"}\NormalTok{)}
\NormalTok{KNNConfMatrix2}
\end{Highlighting}
\end{Shaded}

\begin{verbatim}
## Confusion Matrix and Statistics
## 
##           Reference
## Prediction Abnormal Normal
##   Abnormal       32      5
##   Normal         10     15
##                                           
##                Accuracy : 0.7581          
##                  95% CI : (0.6326, 0.8578)
##     No Information Rate : 0.6774          
##     P-Value [Acc > NIR] : 0.1089          
##                                           
##                   Kappa : 0.4804          
##                                           
##  Mcnemar's Test P-Value : 0.3017          
##                                           
##             Sensitivity : 0.7619          
##             Specificity : 0.7500          
##          Pos Pred Value : 0.8649          
##          Neg Pred Value : 0.6000          
##              Prevalence : 0.6774          
##          Detection Rate : 0.5161          
##    Detection Prevalence : 0.5968          
##       Balanced Accuracy : 0.7560          
##                                           
##        'Positive' Class : Abnormal        
## 
\end{verbatim}

\hypertarget{plot-of-roc-vs-number-of-neighbors-1}{%
\paragraph{Plot of ROC vs Number of
Neighbors}\label{plot-of-roc-vs-number-of-neighbors-1}}

\begin{Shaded}
\begin{Highlighting}[]
\KeywordTok{print}\NormalTok{(}\KeywordTok{ggplot}\NormalTok{(KNNModel2))}
\end{Highlighting}
\end{Shaded}

\begin{center}\includegraphics{NETCapstone_files/figure-latex/unnamed-chunk-46-1} \end{center}

Once again there is very little impact cause by removing the highly
correlated variables, there is only a small increase in the accuracy
from the Original to the Cleaned Dataset.

\hypertarget{neural-network-with-lda-model}{%
\subsection{Neural Network with LDA
Model}\label{neural-network-with-lda-model}}

Artificial Neural Networks {[}NN{]} is a type of mathematical algorithms
that imitates the simulation of networks of biological neurons. It tries
to imatate the human brains neural pathways.

An NN consists of nodes (called neurons) and edges (called synapses).
Input data is transmitted through the weighted synapses to the neurons.
The neurons make calculations that are processed and then passed onto
the next neurons passed to a neuron representing the output (this
implise that end).

NN creates a weighting for connections between neurons. Once all
weightings have been trained, the NN is able to use these connection
make predictions from the input data. NN make use of both forward and
Backpropagation. Backpropagations the set of learning rules used to
guide NN. (Yiu 2019)

\hypertarget{original-dataset-5}{%
\subsubsection{Original Dataset}\label{original-dataset-5}}

\hypertarget{test-and-training-data}{%
\paragraph{Test and Training data}\label{test-and-training-data}}

\begin{Shaded}
\begin{Highlighting}[]
\NormalTok{trainDataLda <-}\StringTok{ }\NormalTok{ldaDataPredict[datSamplingIndex, ]}
\NormalTok{testDataLda <-}\StringTok{ }\NormalTok{ldaDataPredict[}\OperatorTok{-}\NormalTok{datSamplingIndex, ]}
\end{Highlighting}
\end{Shaded}

\hypertarget{model}{%
\paragraph{Model}\label{model}}

\begin{Shaded}
\begin{Highlighting}[]
\NormalTok{NNLDAModel <-}\StringTok{ }\KeywordTok{train}\NormalTok{(class}\OperatorTok{~}\NormalTok{., }
\NormalTok{                 trainDataLda,}
                 \DataTypeTok{method=}\StringTok{"nnet"}\NormalTok{,}
                 \DataTypeTok{metric=}\StringTok{"ROC"}\NormalTok{,}
                 \DataTypeTok{preProcess=}\KeywordTok{c}\NormalTok{(}\StringTok{'center'}\NormalTok{, }\StringTok{'scale'}\NormalTok{),}
                 \DataTypeTok{tuneLength=}\DecValTok{10}\NormalTok{,}
                 \DataTypeTok{trace=}\OtherTok{FALSE}\NormalTok{,}
                 \DataTypeTok{trControl=}\NormalTok{fitControl)}

\NormalTok{NNLdaPred <-}\StringTok{ }\KeywordTok{predict}\NormalTok{(NNLDAModel, testDataLda)}
\NormalTok{NNLdaConfMatrix <-}\StringTok{ }\KeywordTok{confusionMatrix}\NormalTok{(NNLdaPred, }\KeywordTok{as.factor}\NormalTok{(testDataLda}\OperatorTok{$}\NormalTok{class), }\DataTypeTok{positive =} \StringTok{"Abnormal"}\NormalTok{)}
\NormalTok{NNLdaConfMatrix}
\end{Highlighting}
\end{Shaded}

\begin{verbatim}
## Confusion Matrix and Statistics
## 
##           Reference
## Prediction Abnormal Normal
##   Abnormal       35      3
##   Normal          7     17
##                                           
##                Accuracy : 0.8387          
##                  95% CI : (0.7233, 0.9198)
##     No Information Rate : 0.6774          
##     P-Value [Acc > NIR] : 0.003344        
##                                           
##                   Kappa : 0.6493          
##                                           
##  Mcnemar's Test P-Value : 0.342782        
##                                           
##             Sensitivity : 0.8333          
##             Specificity : 0.8500          
##          Pos Pred Value : 0.9211          
##          Neg Pred Value : 0.7083          
##              Prevalence : 0.6774          
##          Detection Rate : 0.5645          
##    Detection Prevalence : 0.6129          
##       Balanced Accuracy : 0.8417          
##                                           
##        'Positive' Class : Abnormal        
## 
\end{verbatim}

\hypertarget{cleaned-dataset-5}{%
\subsubsection{Cleaned Dataset}\label{cleaned-dataset-5}}

\hypertarget{test-and-training-data-1}{%
\paragraph{Test and Training data}\label{test-and-training-data-1}}

\begin{Shaded}
\begin{Highlighting}[]
\NormalTok{trainDataLda2 <-}\StringTok{ }\NormalTok{ldaDataPredict[datSamplingIndex, ]}
\NormalTok{testDataLda2 <-}\StringTok{ }\NormalTok{ldaDataPredict[}\OperatorTok{-}\NormalTok{datSamplingIndex, ]}
\end{Highlighting}
\end{Shaded}

\hypertarget{model-1}{%
\paragraph{Model}\label{model-1}}

\begin{Shaded}
\begin{Highlighting}[]
\NormalTok{NNLDAModel2 <-}\StringTok{ }\KeywordTok{train}\NormalTok{(class}\OperatorTok{~}\NormalTok{., }
\NormalTok{                 trainDataLda2,}
                 \DataTypeTok{method=}\StringTok{"nnet"}\NormalTok{,}
                 \DataTypeTok{metric=}\StringTok{"ROC"}\NormalTok{,}
                 \DataTypeTok{preProcess=}\KeywordTok{c}\NormalTok{(}\StringTok{'center'}\NormalTok{, }\StringTok{'scale'}\NormalTok{),}
                 \DataTypeTok{tuneLength=}\DecValTok{10}\NormalTok{,}
                 \DataTypeTok{trace=}\OtherTok{FALSE}\NormalTok{,}
                 \DataTypeTok{trControl=}\NormalTok{fitControl2)}

\NormalTok{NNLdaPred2 <-}\StringTok{ }\KeywordTok{predict}\NormalTok{(NNLDAModel2, testDataLda2)}
\NormalTok{NNLdaConfMatrix2 <-}\StringTok{ }\KeywordTok{confusionMatrix}\NormalTok{(NNLdaPred2, }\KeywordTok{as.factor}\NormalTok{(testDataLda2}\OperatorTok{$}\NormalTok{class), }\DataTypeTok{positive =} \StringTok{"Abnormal"}\NormalTok{)}
\NormalTok{NNLdaConfMatrix2}
\end{Highlighting}
\end{Shaded}

\begin{verbatim}
## Confusion Matrix and Statistics
## 
##           Reference
## Prediction Abnormal Normal
##   Abnormal       35      3
##   Normal          7     17
##                                           
##                Accuracy : 0.8387          
##                  95% CI : (0.7233, 0.9198)
##     No Information Rate : 0.6774          
##     P-Value [Acc > NIR] : 0.003344        
##                                           
##                   Kappa : 0.6493          
##                                           
##  Mcnemar's Test P-Value : 0.342782        
##                                           
##             Sensitivity : 0.8333          
##             Specificity : 0.8500          
##          Pos Pred Value : 0.9211          
##          Neg Pred Value : 0.7083          
##              Prevalence : 0.6774          
##          Detection Rate : 0.5645          
##    Detection Prevalence : 0.6129          
##       Balanced Accuracy : 0.8417          
##                                           
##        'Positive' Class : Abnormal        
## 
\end{verbatim}

There has been no impact from using the Cleaned Dataset

\hypertarget{neural-network-with-pca-model}{%
\subsection{Neural Network with PCA
Model}\label{neural-network-with-pca-model}}

\hypertarget{original-dataset-6}{%
\subsubsection{Original Dataset}\label{original-dataset-6}}

\begin{Shaded}
\begin{Highlighting}[]
\NormalTok{NNPCAModel <-}\StringTok{ }\KeywordTok{train}\NormalTok{(class}\OperatorTok{~}\NormalTok{.,}
\NormalTok{                        trainData,}
                        \DataTypeTok{method=}\StringTok{"nnet"}\NormalTok{,}
                        \DataTypeTok{metric=}\StringTok{"ROC"}\NormalTok{,}
                        \DataTypeTok{preProcess=}\KeywordTok{c}\NormalTok{(}\StringTok{'center'}\NormalTok{, }\StringTok{'scale'}\NormalTok{, }\StringTok{'pca'}\NormalTok{),}
                        \DataTypeTok{tuneLength=}\DecValTok{10}\NormalTok{,}
                        \DataTypeTok{trace=}\OtherTok{FALSE}\NormalTok{,}
                        \DataTypeTok{trControl=}\NormalTok{fitControl)}

\NormalTok{NNPcaPred <-}\StringTok{ }\KeywordTok{predict}\NormalTok{(NNPCAModel, testData)}
\NormalTok{NNPcaConfMatrix <-}\StringTok{ }\KeywordTok{confusionMatrix}\NormalTok{(NNPcaPred, }\KeywordTok{as.factor}\NormalTok{(testData}\OperatorTok{$}\NormalTok{class), }\DataTypeTok{positive =} \StringTok{"Abnormal"}\NormalTok{)}
\NormalTok{NNPcaConfMatrix}
\end{Highlighting}
\end{Shaded}

\begin{verbatim}
## Confusion Matrix and Statistics
## 
##           Reference
## Prediction Abnormal Normal
##   Abnormal       36      5
##   Normal          6     15
##                                          
##                Accuracy : 0.8226         
##                  95% CI : (0.7047, 0.908)
##     No Information Rate : 0.6774         
##     P-Value [Acc > NIR] : 0.008074       
##                                          
##                   Kappa : 0.5993         
##                                          
##  Mcnemar's Test P-Value : 1.000000       
##                                          
##             Sensitivity : 0.8571         
##             Specificity : 0.7500         
##          Pos Pred Value : 0.8780         
##          Neg Pred Value : 0.7143         
##              Prevalence : 0.6774         
##          Detection Rate : 0.5806         
##    Detection Prevalence : 0.6613         
##       Balanced Accuracy : 0.8036         
##                                          
##        'Positive' Class : Abnormal       
## 
\end{verbatim}

\hypertarget{plot-of-variable-importance-4}{%
\paragraph{Plot of Variable
Importance}\label{plot-of-variable-importance-4}}

\begin{Shaded}
\begin{Highlighting}[]
\KeywordTok{varImp}\NormalTok{(NNPCAModel)}
\end{Highlighting}
\end{Shaded}

\begin{verbatim}
## nnet variable importance
## 
##      Overall
## PC4 100.0000
## PC5  41.1974
## PC1  36.8820
## PC2   0.7919
## PC3   0.0000
\end{verbatim}

\begin{Shaded}
\begin{Highlighting}[]
\KeywordTok{plot}\NormalTok{(}\KeywordTok{varImp}\NormalTok{(NNPCAModel), }\DataTypeTok{main=}\StringTok{"Ranking Of Importance"}\NormalTok{)}
\end{Highlighting}
\end{Shaded}

\includegraphics{NETCapstone_files/figure-latex/unnamed-chunk-53-1.pdf}

\hypertarget{cleaned-dataset-6}{%
\subsubsection{Cleaned Dataset}\label{cleaned-dataset-6}}

\begin{Shaded}
\begin{Highlighting}[]
\NormalTok{NNPCAModel2 <-}\StringTok{ }\KeywordTok{train}\NormalTok{(class}\OperatorTok{~}\NormalTok{.,}
\NormalTok{                        trainData2,}
                        \DataTypeTok{method=}\StringTok{"nnet"}\NormalTok{,}
                        \DataTypeTok{metric=}\StringTok{"ROC"}\NormalTok{,}
                        \DataTypeTok{preProcess=}\KeywordTok{c}\NormalTok{(}\StringTok{'center'}\NormalTok{, }\StringTok{'scale'}\NormalTok{, }\StringTok{'pca'}\NormalTok{),}
                        \DataTypeTok{tuneLength=}\DecValTok{10}\NormalTok{,}
                        \DataTypeTok{trace=}\OtherTok{FALSE}\NormalTok{,}
                        \DataTypeTok{trControl=}\NormalTok{fitControl2)}

\NormalTok{NNPcaPred2 <-}\StringTok{ }\KeywordTok{predict}\NormalTok{(NNPCAModel2, testData2)}
\NormalTok{NNPcaConfMatrix2 <-}\StringTok{ }\KeywordTok{confusionMatrix}\NormalTok{(NNPcaPred2, }\KeywordTok{as.factor}\NormalTok{(testData2}\OperatorTok{$}\NormalTok{class), }\DataTypeTok{positive =} \StringTok{"Abnormal"}\NormalTok{)}
\NormalTok{NNPcaConfMatrix2}
\end{Highlighting}
\end{Shaded}

\begin{verbatim}
## Confusion Matrix and Statistics
## 
##           Reference
## Prediction Abnormal Normal
##   Abnormal       35      5
##   Normal          7     15
##                                           
##                Accuracy : 0.8065          
##                  95% CI : (0.6863, 0.8958)
##     No Information Rate : 0.6774          
##     P-Value [Acc > NIR] : 0.01765         
##                                           
##                   Kappa : 0.5684          
##                                           
##  Mcnemar's Test P-Value : 0.77283         
##                                           
##             Sensitivity : 0.8333          
##             Specificity : 0.7500          
##          Pos Pred Value : 0.8750          
##          Neg Pred Value : 0.6818          
##              Prevalence : 0.6774          
##          Detection Rate : 0.5645          
##    Detection Prevalence : 0.6452          
##       Balanced Accuracy : 0.7917          
##                                           
##        'Positive' Class : Abnormal        
## 
\end{verbatim}

There has been a negative impact from using the Cleaned Dataset on the
NN using PCA. This can be seen via the decrease in accuracy from 0.8226
to 0.8065, from the Original to the Clean Dataset respectively.

\hypertarget{plot-of-variable-importance-5}{%
\paragraph{Plot of Variable
Importance}\label{plot-of-variable-importance-5}}

\begin{Shaded}
\begin{Highlighting}[]
\KeywordTok{varImp}\NormalTok{(NNPCAModel2)}
\end{Highlighting}
\end{Shaded}

\begin{verbatim}
## nnet variable importance
## 
##     Overall
## PC4  100.00
## PC1   59.60
## PC3   55.02
## PC2    0.00
\end{verbatim}

\begin{Shaded}
\begin{Highlighting}[]
\KeywordTok{plot}\NormalTok{(}\KeywordTok{varImp}\NormalTok{(NNPCAModel2), }\DataTypeTok{main=}\StringTok{"Ranking Of Importance"}\NormalTok{)}
\end{Highlighting}
\end{Shaded}

\includegraphics{NETCapstone_files/figure-latex/unnamed-chunk-56-1.pdf}

\newpage

\hypertarget{results}{%
\section{Results}\label{results}}

\hypertarget{original-dataset-7}{%
\subsection{Original DataSet}\label{original-dataset-7}}

\hypertarget{model-comparisions}{%
\subsubsection{Model Comparisions}\label{model-comparisions}}

\begin{Shaded}
\begin{Highlighting}[]
\NormalTok{models <-}\StringTok{ }\KeywordTok{list}\NormalTok{(}\DataTypeTok{NaiveBayes =}\NormalTok{ NaiveBayesModel, }
                    \DataTypeTok{LogReg =}\NormalTok{ LogRegModel,}
                    \DataTypeTok{RandomForest =}\NormalTok{ RandomforestModel,}
                    \DataTypeTok{KNN =}\NormalTok{ KNNModel,}
                    \DataTypeTok{NeuralPCA =}\NormalTok{ NNPCAModel,}
                    \DataTypeTok{NeuralLDA =}\NormalTok{ NNLDAModel)   }

\NormalTok{modelsResults <-}\StringTok{ }\KeywordTok{resamples}\NormalTok{(models)}
\KeywordTok{summary}\NormalTok{(modelsResults)}
\end{Highlighting}
\end{Shaded}

\begin{verbatim}
## 
## Call:
## summary.resamples(object = modelsResults)
## 
## Models: NaiveBayes, LogReg, RandomForest, KNN, NeuralPCA, NeuralLDA 
## Number of resamples: 20 
## 
## ROC 
##                   Min.   1st Qu.    Median      Mean   3rd Qu. Max. NA's
## NaiveBayes   0.7500000 0.8437500 0.8750000 0.8727431 0.9375000    1    0
## LogReg       0.7812500 0.9019097 0.9375000 0.9303819 1.0000000    1    0
## RandomForest 0.7500000 0.8750000 0.9270833 0.9119792 0.9722222    1    0
## KNN          0.6250000 0.8750000 0.9218750 0.8969618 0.9696181    1    0
## NeuralPCA    0.7777778 0.9140625 0.9375000 0.9361111 0.9696181    1    0
## NeuralLDA    0.7500000 0.8567708 0.9062500 0.9100694 0.9696181    1    0
## 
## Sens 
##                   Min.   1st Qu.    Median      Mean   3rd Qu.  Max. NA's
## NaiveBayes   0.5555556 0.6666667 0.7500000 0.7465278 0.7777778 0.875    0
## LogReg       0.6250000 0.7708333 0.8819444 0.8631944 1.0000000 1.000    0
## RandomForest 0.7500000 0.8506944 0.8819444 0.8854167 1.0000000 1.000    0
## KNN          0.6250000 0.7708333 0.8750000 0.8659722 1.0000000 1.000    0
## NeuralPCA    0.5000000 0.7777778 0.8819444 0.8763889 1.0000000 1.000    0
## NeuralLDA    0.7500000 0.8506944 0.8888889 0.8819444 0.9166667 1.000    0
## 
## Spec 
##              Min. 1st Qu. Median   Mean 3rd Qu. Max. NA's
## NaiveBayes   0.50  0.7500  0.750 0.8375  1.0000    1    0
## LogReg       0.25  0.5000  0.875 0.7875  1.0000    1    0
## RandomForest 0.50  0.5000  0.750 0.7375  1.0000    1    0
## KNN          0.25  0.5000  0.750 0.7250  0.8125    1    0
## NeuralPCA    0.50  0.7500  0.750 0.8250  1.0000    1    0
## NeuralLDA    0.00  0.6875  0.750 0.7375  1.0000    1    0
\end{verbatim}

\begin{Shaded}
\begin{Highlighting}[]
\KeywordTok{print}\NormalTok{(}\KeywordTok{bwplot}\NormalTok{(modelsResults, }\DataTypeTok{metric=}\StringTok{"ROC"}\NormalTok{,}\DataTypeTok{main =} \StringTok{"Variablities"}\NormalTok{))}
\end{Highlighting}
\end{Shaded}

\begin{center}\includegraphics{NETCapstone_files/figure-latex/unnamed-chunk-58-1} \end{center}

As we can observe from the plot above, Three models, Naive Bayes ,
Neural LDA and Random Forest have great variability.

The Receiver Operating characteristic Curve {[}ROC{]} is a graph that
shows the performance of a classification model at all classification
thresholds. AUC is the metric measure of the ROC curve of each model.
This metric is independent of any threshold (Narkhede, n.d.).

The NN LDA model managed to achieve a great Area Under the ROC Curve
{[}AUC{]} but has a high variability. Secondly the NN PCA managed to
also achieve a good Area Under the ROC Curve but with less variability
that the NN LDA which indicates that it is performing better.\\

\begin{Shaded}
\begin{Highlighting}[]
\NormalTok{confMatrixs <-}\StringTok{ }\KeywordTok{list}\NormalTok{(}
  \DataTypeTok{NaiveBayes =}\NormalTok{ NBConfMatrix, }
  \DataTypeTok{LogReg =}\NormalTok{ LogRegConfMatrix,}
  \DataTypeTok{RandomForest =}\NormalTok{ RFConfMatrix,}
  \DataTypeTok{KNN =}\NormalTok{ KNNConfMatrix,}
  \DataTypeTok{NeuralPCA =}\NormalTok{ NNPcaConfMatrix,}
  \DataTypeTok{NeuralLDA =}\NormalTok{ NNPcaConfMatrix)   }

\NormalTok{ConfMatrixResults <-}\StringTok{ }\KeywordTok{sapply}\NormalTok{(confMatrixs, }\ControlFlowTok{function}\NormalTok{(x) x}\OperatorTok{$}\NormalTok{byClass)}
\NormalTok{ConfMatrixResults }\OperatorTok\StringTok{ }\NormalTok{knitr}\OperatorTok{::}\KeywordTok{kable}\NormalTok{()}
\end{Highlighting}
\end{Shaded}

\begin{tabular}{l|r|r|r|r|r|r}
\hline
  & NaiveBayes & LogReg & RandomForest & KNN & NeuralPCA & NeuralLDA\\
\hline
Sensitivity & 0.5952381 & 0.8571429 & 0.8333333 & 0.7619048 & 0.8571429 & 0.8571429\\
\hline
Specificity & 0.9000000 & 0.7500000 & 0.7500000 & 0.7000000 & 0.7500000 & 0.7500000\\
\hline
Pos Pred Value & 0.9259259 & 0.8780488 & 0.8750000 & 0.8421053 & 0.8780488 & 0.8780488\\
\hline
Neg Pred Value & 0.5142857 & 0.7142857 & 0.6818182 & 0.5833333 & 0.7142857 & 0.7142857\\
\hline
Precision & 0.9259259 & 0.8780488 & 0.8750000 & 0.8421053 & 0.8780488 & 0.8780488\\
\hline
Recall & 0.5952381 & 0.8571429 & 0.8333333 & 0.7619048 & 0.8571429 & 0.8571429\\
\hline
F1 & 0.7246377 & 0.8674699 & 0.8536585 & 0.8000000 & 0.8674699 & 0.8674699\\
\hline
Prevalence & 0.6774194 & 0.6774194 & 0.6774194 & 0.6774194 & 0.6774194 & 0.6774194\\
\hline
Detection Rate & 0.4032258 & 0.5806452 & 0.5645161 & 0.5161290 & 0.5806452 & 0.5806452\\
\hline
Detection Prevalence & 0.4354839 & 0.6612903 & 0.6451613 & 0.6129032 & 0.6612903 & 0.6612903\\
\hline
Balanced Accuracy & 0.7476190 & 0.8035714 & 0.7916667 & 0.7309524 & 0.8035714 & 0.8035714\\
\hline
\end{tabular}

\hypertarget{cleaned-dataset-7}{%
\subsection{Cleaned DataSet}\label{cleaned-dataset-7}}

\hypertarget{model-comparisions-1}{%
\subsubsection{Model Comparisions}\label{model-comparisions-1}}

\begin{Shaded}
\begin{Highlighting}[]
\NormalTok{models2 <-}\StringTok{ }\KeywordTok{list}\NormalTok{(}\DataTypeTok{NaiveBayes =}\NormalTok{ NaiveBayesModel2, }
                    \DataTypeTok{LogReg =}\NormalTok{ LogRegModel2,}
                    \DataTypeTok{RandomForest =}\NormalTok{ RandomforestModel2,}
                    \DataTypeTok{KNN =}\NormalTok{ KNNModel2,}
                    \DataTypeTok{NeuralPCA =}\NormalTok{ NNPCAModel2,}
                    \DataTypeTok{NeuralLDA =}\NormalTok{ NNLDAModel2)   }

\NormalTok{modelsResults2 <-}\StringTok{ }\KeywordTok{resamples}\NormalTok{(models2)}
\KeywordTok{summary}\NormalTok{(modelsResults2)}
\end{Highlighting}
\end{Shaded}

\begin{verbatim}
## 
## Call:
## summary.resamples(object = modelsResults2)
## 
## Models: NaiveBayes, LogReg, RandomForest, KNN, NeuralPCA, NeuralLDA 
## Number of resamples: 20 
## 
## ROC 
##                   Min.   1st Qu.    Median      Mean  3rd Qu. Max. NA's
## NaiveBayes   0.6562500 0.8107639 0.8888889 0.8736111 0.937500    1    0
## LogReg       0.8611111 0.8888889 0.9409722 0.9371528 1.000000    1    0
## RandomForest 0.7500000 0.8750000 0.9114583 0.9211806 1.000000    1    0
## KNN          0.7812500 0.8576389 0.9409722 0.9259549 0.984809    1    0
## NeuralPCA    0.6250000 0.9062500 0.9565972 0.9305556 1.000000    1    0
## NeuralLDA    0.7222222 0.8854167 0.9270833 0.9208333 1.000000    1    0
## 
## Sens 
##                   Min.   1st Qu.    Median      Mean   3rd Qu. Max. NA's
## NaiveBayes   0.3750000 0.6250000 0.7500000 0.7409722 0.8784722    1    0
## LogReg       0.5000000 0.7777778 0.8819444 0.8743056 1.0000000    1    0
## RandomForest 0.6250000 0.7777778 0.8819444 0.8798611 1.0000000    1    0
## KNN          0.7500000 0.8506944 0.8888889 0.9048611 1.0000000    1    0
## NeuralPCA    0.7500000 0.8750000 0.9444444 0.9229167 1.0000000    1    0
## NeuralLDA    0.6666667 0.8437500 0.8888889 0.8930556 1.0000000    1    0
## 
## Spec 
##              Min. 1st Qu. Median   Mean 3rd Qu. Max. NA's
## NaiveBayes   0.50  0.7500   1.00 0.8875  1.0000    1    0
## LogReg       0.50  0.7500   0.75 0.8000  1.0000    1    0
## RandomForest 0.50  0.5000   0.75 0.7625  1.0000    1    0
## KNN          0.50  0.6875   0.75 0.7500  0.8125    1    0
## NeuralPCA    0.00  0.7500   0.75 0.7500  1.0000    1    0
## NeuralLDA    0.25  0.5000   0.75 0.7375  1.0000    1    0
\end{verbatim}

\begin{Shaded}
\begin{Highlighting}[]
\KeywordTok{bwplot}\NormalTok{(modelsResults2, }\DataTypeTok{metric=}\StringTok{"ROC"}\NormalTok{,}\DataTypeTok{main =} \StringTok{"Variablities"}\NormalTok{)}
\end{Highlighting}
\end{Shaded}

\begin{center}\includegraphics{NETCapstone_files/figure-latex/unnamed-chunk-61-1} \end{center}

Comparing the Cleaned Dataset to the original it is important to note
although the had been no real differences seen when running the models.
There is a clear difference when comparing the variability plots. It is
clear that by removing the highly correlated variable reduced some of
the variability in the ROC for particular Models which is good, but also
has seemed to increase variability in some of the other models

From the plot above it is clear that the models now experiencing the
greatest variability are Naive Bayes and Neural LDA. The models now with
the best ROC are the LogReg and Neural PCA.

\begin{Shaded}
\begin{Highlighting}[]
\NormalTok{confMatrixs2 <-}\StringTok{ }\KeywordTok{list}\NormalTok{(}
  \DataTypeTok{NaiveBayes =}\NormalTok{ NBConfMatrix2, }
  \DataTypeTok{LogReg =}\NormalTok{ LogRegConfMatrix2,}
  \DataTypeTok{RandomForest =}\NormalTok{ RFConfMatrix2,}
  \DataTypeTok{KNN =}\NormalTok{ KNNConfMatrix2,}
  \DataTypeTok{NeuralPCA =}\NormalTok{ NNPcaConfMatrix2,}
  \DataTypeTok{NeuralLDA =}\NormalTok{ NNPcaConfMatrix2)   }

\NormalTok{ConfMatrixResults2 <-}\StringTok{ }\KeywordTok{sapply}\NormalTok{(confMatrixs2, }\ControlFlowTok{function}\NormalTok{(x) x}\OperatorTok{$}\NormalTok{byClass)}
\NormalTok{ConfMatrixResults2 }\OperatorTok\StringTok{ }\NormalTok{knitr}\OperatorTok{::}\KeywordTok{kable}\NormalTok{()}
\end{Highlighting}
\end{Shaded}

\begin{tabular}{l|r|r|r|r|r|r}
\hline
  & NaiveBayes & LogReg & RandomForest & KNN & NeuralPCA & NeuralLDA\\
\hline
Sensitivity & 0.6428571 & 0.8571429 & 0.8333333 & 0.7619048 & 0.8333333 & 0.8333333\\
\hline
Specificity & 0.9500000 & 0.7500000 & 0.7000000 & 0.7500000 & 0.7500000 & 0.7500000\\
\hline
Pos Pred Value & 0.9642857 & 0.8780488 & 0.8536585 & 0.8648649 & 0.8750000 & 0.8750000\\
\hline
Neg Pred Value & 0.5588235 & 0.7142857 & 0.6666667 & 0.6000000 & 0.6818182 & 0.6818182\\
\hline
Precision & 0.9642857 & 0.8780488 & 0.8536585 & 0.8648649 & 0.8750000 & 0.8750000\\
\hline
Recall & 0.6428571 & 0.8571429 & 0.8333333 & 0.7619048 & 0.8333333 & 0.8333333\\
\hline
F1 & 0.7714286 & 0.8674699 & 0.8433735 & 0.8101266 & 0.8536585 & 0.8536585\\
\hline
Prevalence & 0.6774194 & 0.6774194 & 0.6774194 & 0.6774194 & 0.6774194 & 0.6774194\\
\hline
Detection Rate & 0.4354839 & 0.5806452 & 0.5645161 & 0.5161290 & 0.5645161 & 0.5645161\\
\hline
Detection Prevalence & 0.4516129 & 0.6612903 & 0.6612903 & 0.5967742 & 0.6451613 & 0.6451613\\
\hline
Balanced Accuracy & 0.7964286 & 0.8035714 & 0.7666667 & 0.7559524 & 0.7916667 & 0.7916667\\
\hline
\end{tabular}

\newpage

\hypertarget{discussion}{%
\subsection{Discussion}\label{discussion}}

The following metrics will be compared in this discussion:

\begin{itemize}
\item
  Accuracy: It is the number of correct predictions made divided by the
  total number of predictions made, multiplied by 100 to turn it into a
  percentage.
\item
  Precision is the number of True Positives divided by the number of
  True Positives and False Positives. Or can be referred to as the
  number of positive predictions divided by the total number of positive
  class values predicted. This is refered to as the Positive Predictive
  Value (PPV). A low precision can indicate a large number of False
  Positives.
\item
  Sensitivity or True Positive Rate {[}recall{]} is the number of True
  Positives divided by the number of True Positives and the number of
  False Negatives. Or can be seen as the number of positive predictions
  divided by the number of positive class values in the test data.
  Recall can be thought of as a measure of a classifiers completeness. A
  low recall indicates many False Negatives.
\item
  The F1 Score is calculated from: 2 x ((precision x recall) /
  (precision + recall)). It is also called the F Score or the F Measure.
  The F1 score conveys the balance between the precision and the recall.
\end{itemize}

\hypertarget{original-dataset-8}{%
\subsubsection{Original Dataset}\label{original-dataset-8}}

From within the original datset :

The best results for sensitivity (detection of Abnormal features of
orthopedic patients) is NN Model with the PCA model which also has the
best F1 score. It should be note the NN Model with PCA model and the NN
Model with LDA have the same Sensitivity score. As the PCA model has the
best F1 store it has been rated as best overall.

\begin{Shaded}
\begin{Highlighting}[]
\NormalTok{MaxConfMatrix <-}\StringTok{ }\KeywordTok{apply}\NormalTok{(ConfMatrixResults, }\DecValTok{1}\NormalTok{, which.is.max)}
\NormalTok{OutputReport <-}\StringTok{ }\KeywordTok{data.frame}\NormalTok{(}\DataTypeTok{metric=}\KeywordTok{names}\NormalTok{(MaxConfMatrix), }
                            \DataTypeTok{bestModel=}\KeywordTok{colnames}\NormalTok{(ConfMatrixResults)[MaxConfMatrix],}
                            \DataTypeTok{value=}\KeywordTok{mapply}\NormalTok{(}\ControlFlowTok{function}\NormalTok{(x,y) \{ConfMatrixResults[x,y]\}, }
                                         \KeywordTok{names}\NormalTok{(MaxConfMatrix), }
\NormalTok{                                         MaxConfMatrix))}
\KeywordTok{rownames}\NormalTok{(OutputReport) <-}\StringTok{ }\OtherTok{NULL}
\NormalTok{OutputReport}
\end{Highlighting}
\end{Shaded}

\begin{verbatim}
##                  metric  bestModel     value
## 1           Sensitivity  NeuralLDA 0.8571429
## 2           Specificity NaiveBayes 0.9000000
## 3        Pos Pred Value NaiveBayes 0.9259259
## 4        Neg Pred Value     LogReg 0.7142857
## 5             Precision NaiveBayes 0.9259259
## 6                Recall     LogReg 0.8571429
## 7                    F1  NeuralPCA 0.8674699
## 8            Prevalence  NeuralPCA 0.6774194
## 9        Detection Rate  NeuralLDA 0.5806452
## 10 Detection Prevalence  NeuralLDA 0.6612903
## 11    Balanced Accuracy     LogReg 0.8035714
\end{verbatim}

\hypertarget{cleaned-dataset-8}{%
\subsubsection{Cleaned Dataset}\label{cleaned-dataset-8}}

From within the Cleaned Dataset datset :

The best results for sensitivity (detection of Abnormal features of
orthopedic patients) is the LogReg which also has the best F1 score.

\begin{Shaded}
\begin{Highlighting}[]
\NormalTok{MaxConfMatrix2 <-}\StringTok{ }\KeywordTok{apply}\NormalTok{(ConfMatrixResults2, }\DecValTok{1}\NormalTok{, which.is.max)}
\NormalTok{OutputReport2 <-}\StringTok{ }\KeywordTok{data.frame}\NormalTok{(}\DataTypeTok{metric=}\KeywordTok{names}\NormalTok{(MaxConfMatrix2), }
                            \DataTypeTok{bestModel=}\KeywordTok{colnames}\NormalTok{(ConfMatrixResults2)[MaxConfMatrix2],}
                            \DataTypeTok{value=}\KeywordTok{mapply}\NormalTok{(}\ControlFlowTok{function}\NormalTok{(x,y) \{ConfMatrixResults2[x,y]\}, }
                                         \KeywordTok{names}\NormalTok{(MaxConfMatrix2), }
\NormalTok{                                         MaxConfMatrix2))}
\KeywordTok{rownames}\NormalTok{(OutputReport2) <-}\StringTok{ }\OtherTok{NULL}
\NormalTok{OutputReport2}
\end{Highlighting}
\end{Shaded}

\begin{verbatim}
##                  metric    bestModel     value
## 1           Sensitivity       LogReg 0.8571429
## 2           Specificity   NaiveBayes 0.9500000
## 3        Pos Pred Value   NaiveBayes 0.9642857
## 4        Neg Pred Value       LogReg 0.7142857
## 5             Precision   NaiveBayes 0.9642857
## 6                Recall       LogReg 0.8571429
## 7                    F1       LogReg 0.8674699
## 8            Prevalence RandomForest 0.6774194
## 9        Detection Rate       LogReg 0.5806452
## 10 Detection Prevalence       LogReg 0.6612903
## 11    Balanced Accuracy       LogReg 0.8035714
\end{verbatim}

\newpage

\hypertarget{conclusion}{%
\section{Conclusion}\label{conclusion}}

This paper treats the Biomechanical features of orthopedic patients
diagnosis problem as a pattern classification problem. This report
investigates several machine learning models the optimal model is
selected by selecting a high accuracy level combined with a low rate of
false-negatives (meaning high sensitivity).

From this it can be concluded that from the Original Dataset the NN
Model with the PCA Model had the optimal results for F1 (0.8674699),
Sensitivity (0.8571429) and Balanced Accuracy (0.8035714).

Secondly it can also be concluded that from the Cleaned Dataset the
LogReg Model had the optimal results for F1 (0.8674699), Sensitivity
(0.8571429) and Balanced Accuracy (0.8035714).

To draw a conclusion in good machine learning practise it is always best
to remove highly correlated variable. Therefore one should theoretical
use the Cleaned Dataset with the LogReg model when making predictions.

\hypertarget{future-work}{%
\subsection{Future work}\label{future-work}}

Future work on this particular project should focus on checking whether
the predictions made by the models that had the optimal results actually
give outputs that can be trusted. This could be done by supplying the
Models with unfinished data and comparing the models results to verified
data.

Also future work should focus on trying to improve the models accuracy
and reducing the variability further.

\newpage

\hypertarget{appendix}{%
\section{Appendix}\label{appendix}}

\hypertarget{environment}{%
\subsection{Environment}\label{environment}}

\begin{Shaded}
\begin{Highlighting}[]
\KeywordTok{print}\NormalTok{(}\StringTok{"Operating System:"}\NormalTok{)}
\end{Highlighting}
\end{Shaded}

\begin{verbatim}
## [1] "Operating System:"
\end{verbatim}

\begin{Shaded}
\begin{Highlighting}[]
\NormalTok{version}
\end{Highlighting}
\end{Shaded}

\begin{verbatim}
##                _                           
## platform       x86_64-apple-darwin17.0     
## arch           x86_64                      
## os             darwin17.0                  
## system         x86_64, darwin17.0          
## status                                     
## major          4                           
## minor          0.3                         
## year           2020                        
## month          10                          
## day            10                          
## svn rev        79318                       
## language       R                           
## version.string R version 4.0.3 (2020-10-10)
## nickname       Bunny-Wunnies Freak Out
\end{verbatim}

\hypertarget{references}{%
\section*{References}\label{references}}
\addcontentsline{toc}{section}{References}

\hypertarget{refs}{}
\leavevmode\hypertarget{ref-KNN2}{}%
Harrison, Onel. n.d. ``Machine Learning Basics with the K-Nearest
Neighbors Algorithm.''
\url{https://towardsdatascience.com/machine-learning-basics-with-the-k-nearest-neighbors-algorithm-6a6e71d01761}.

\leavevmode\hypertarget{ref-Data}{}%
Learning, UCI Machine. n.d. ``Biomechanical Features of Orthopedic
Patients.''
\url{https://www.kaggle.com/uciml/biomechanical-features-of-orthopedic-patients}.

\leavevmode\hypertarget{ref-Spody}{}%
Moore, Kristeen. 2018. ``Spondylolisthesis.''
\url{https://www.healthline.com/health/spondylolisthesis}.

\leavevmode\hypertarget{ref-NaiveBay}{}%
``Naive Bayesian.'' n.d.
\url{https://www.saedsayad.com/naive_bayesian.htm}.

\leavevmode\hypertarget{ref-ROCCurve}{}%
Narkhede, Sarang. n.d. ``Understanding Auc - Roc Curve.''
\url{https://towardsdatascience.com/understanding-auc-roc-curve-68b2303cc9c5}.

\leavevmode\hypertarget{ref-LDA2}{}%
Peixeiro, Marco. n.d. ``The Complete Guide to Classification in
Python.''
\url{https://towardsdatascience.com/the-complete-guide-to-classification-in-python-b0e34c92e455}.

\leavevmode\hypertarget{ref-RForest}{}%
Pramoditha, Rukshan. n.d. ``Random Forests --- an Ensemble of Decision
Trees.''
\url{https://towardsdatascience.com/random-forests-an-ensemble-of-decision-trees-37a003084c6c}.

\leavevmode\hypertarget{ref-PCA}{}%
ProjectPro. n.d. ``Principal Component Analysis Tutorial.'' Accessed
2020.
\url{https://www.dezyre.com/data-science-in-python-tutorial/principal-component-analysis-tutorial\#:~:text=The\%20main\%20idea\%20of\%20principal,up\%20to\%20the\%20maximum\%20extent.\&text=As\%20a\%20layman\%2C\%20it\%20is\%20a\%20method\%20of\%20summarizing\%20data.}

\leavevmode\hypertarget{ref-Hernia}{}%
Staff, Mayo Clinic. n.d. ``Herniated Disk.''
\url{https://www.mayoclinic.org/diseases-conditions/herniated-disk/symptoms-causes/syc-20354095}.

\leavevmode\hypertarget{ref-Cor}{}%
University, Andrews. n.d. ``Applied Statistics - Lesson 5.''
\url{https://www.andrews.edu/~calkins/math/edrm611/edrm05.htm\#:~:text=Correlation\%20coefficients\%20whose\%20magnitude\%20are,can\%20be\%20considered\%20highly\%20correlated.}

\leavevmode\hypertarget{ref-NNs}{}%
Yiu, Tony. 2019. ``Understanding Neural Networks.''
\url{https://towardsdatascience.com/understanding-neural-networks-19020b758230}.

\end{document}
